\documentclass{article}
\usepackage{amsmath,amsthm,amsfonts}  
\usepackage{amssymb}                      % AMS Math
\usepackage{thmtools}                                       % Theorem Tools
\usepackage{bm}        
\usepackage{tabto}   
\newtheorem{theorem}{Theorem}                                       % Bold Math

\title{MATH1081 notes}
\author{Nira (z5417727)}
\date{September 17th, 2022}
\begin{document}
\maketitle 
\tableofcontents
\newpage
\section{Topic 1}
\subsection{Introduction}
1. addition, multiplication, division and subtraction
\newline
2. Mainly dealing with finite sets

\newpage
\subsection{Sets and subsets}
\boxed{ $\text{A set is a well defined collection of distinct objects}$ }
\newline
\newline
Example: $S = \{1, a, 3\}, A =\{\Pi, 1\}.$
\newline
1. $e \notin A$; it is not in A
\newline
2. For example, if A is a set of all integers; $\{ \text{all even integers} \}$
    = $\{n \in \mathbb{R} | \text{n is even}\}$.
\newline
3. We can remove superfluos items (elements that occur more than one).
$A = \{ 1, 2, 3, 3 \}$ where $3$ can be removed.
\newline
\newline
\newline
Example:
\newline
$A = \{ 1, 2, 3 \},
B = \{2, 3, 1 \}, 
C = \{1, 2, 3, 3 \},
D = \{ 1 , 3 \}.$
\newline
Here, D is a proper subset of A, B, C; A, B, C are supersets of D.
\newline
\boxed {$$\subseteq: \text{Subset (proper subset)}, $$
$$ \supseteq: \text{Superset}.$$}
\newline
\newline
1. To prove if a set is a proper subset; do the following:
\newline
For example, if $D \in A$, then check if $e \in D$
\newline
If $e \in D,$ then $e \in A$.
Thus, it would be a proper subset (here, e is just an element).
\newline
\newline
2. To prove that two sets are equal;
\newline
For example, if A = B, prove:
\newline
i) $A \subseteq B$; if an element is in A, then the element is in B.
\newline
ii) $B \subseteq A$; if an element is in B, then the element is in A.
\newpage
\subsection{Power Sets and Stability}
Subsets of $A = \{1, 2, 3\} $:
\newline
1. Could throw everything out to get empty set $\Phi$,
\newline
2. One element each: $\{ 1 \}, \{ 2 \}, \{ 3 \}$,
\newline
3. Two elements: $\{ 1, 2 \}, \{ 2, 3 \}, \{ 1, 3 \}$,
\newline
4. Set itself: $A$.
\newline
The set containing 1, 2, 3, 4 is called the powerset of A.
\newline
\newline
Given $
A = \{1, 2, 3\}, B = \{1, 2, 3, 3\}, C = \{1, 3\}, D = \{1, 3\}$, where \newline
$A = B$, $C \subseteq A, B$ and $D \not \subseteq A, B, C$.
\newline
1. size of A = 3, B = 3, C = 2, D = 2.
\newline
[Exercise with A  = {0, 1, {0, 1}}, B done in word].
\newpage
\subsection{Set Operations}
Boolean Operators ("not" operation in programming): 
\newline
1. Complement:
\newline
Let there be a set A in U (A: all of the people in the video, U: universal set of everyone in the world, $A^c =$ complement of A).
\newline
\newline
\boxed { $$A^c = \{ x \in U | x \not \in A \}.$$}
\newline
\newline
2. Intersecting ("and" operation in programming):
\newline
If there is $A, B$, intersecting, 
\newline
\newline
\boxed{ $$A \cap B = \{ x \in A | x \in B\}.$$}
\newline
\newline
3. Union ("or" operation in programming):
If there is $A, B$, A or B is:
\newline
\newline
\boxed{ $$A \cup B = \{ x \in U | x \in A \text { or } x \in B\}.$$}
\newline
\newline
4. Difference:
If there is $A, B$, intersecting, 
\newline
\newline
\boxed{ $$A - B = \{ x \in A | x \in B \}.$$}
\newline
\newline
[examples in word doc]
\newpage
\subsection{The Inclusion-Exclusion Principle}
[example in Word]
\newline
\boxed{ $$|A \cup B| = |A| + |B| - | A \cap B|$$.}
\newline
\newline
For three elements, 
\newline
\newline
\boxed{ $$|A \cup B \cup C| = |A| + |B| + |C| - | A \cap B| - |A \cap C| - |B \cap C| + |A \cap B \cap C|$$.}
\newline
\newline
[example in word]
\newpage
\subsection{Sets Proofs}
[proof question in word]
\newline
Hints for proofs:
\newline
1. To prove that $S \subseteq T$, we can assume that $x \in S$ and show that $x \in T$.
\newline
2. 
To prove that $S = T$, we can show that $S \subseteq T$ and $T \subseteq S$.
\newline
\newline
Scaffold: 
\newline
\boxed{ \text{Proof: Suppose that \dots 
\newline
\dots (proof)
\newline
we see that/ it follows \dots (conclusion) (end with shaded box to indicate end of proof.)}}
\newline
\newline
Note that the "Suppose that" part of the proof is usually whatever the if statment mentions.
\newline
\newline
For example, if the question is "Prove that if $A \cap B = A$, then $A \cup B = B$, then the proof starts like this:
\newline
\boxed{\underline{Proof}: \text{ Suppose that } A \cap B = A. }
\newline
For questions like "is this statement true", there are two ways to approach the question:
\newline
1. If the statement is true (if you think it is true), then prove it.
\newline
2. If the statement is false, then give a counter-example that proves it false. 
\newline
[examples in word]
\newpage
\subsection{Laws of Set Algebra}
\underline{ Laws of Set Algebra}
\newline
$$1\text{. } A \cap B = B \cap A: \text{ Commutative Law}.$$
$$2\text{. } A \cap \text{(} B \cap C \text{)} = \text{(} A \cap B \text{)} \cap C: \text{ Associative Law}.$$
$$3\text{. } A \cap \text{(} B \cap C \text{)} = \text{(} A \cap B \text{)} \cup \text{(} A \cap C \text{)}: \text{ Distributive Law}.$$
$$4\text{. } A \cap \text{(} A \cap B \text{)} = A: \text{ Absorption Law}.$$
$$5\text{. } A \cap U = U \cap A = A: \text{ Identity Law}.$$
$$6\text{. } A \cap A = A: \text{ Idempotent Law}.$$
$$7\text{. } (A^c)^c = A: \text{ Double Complement Law}.$$
$$8\text{. } A \cap \emptyset = \emptyset \cap A = \emptyset: \text{ Domination Law}.$$
$$9\text{. } A \cap A^c = \emptyset: \text{ Intersection with Complement Law}.$$
$$10\text{. } (A \cup B)^c = A^c \cap B^c: \text{ De Moirve's Law}.$$
The intersection can be swapped with the union to form another law (like, $A \cup B = B \cup A$ swapped as $A \cap B = B \cap A.$). Similarily, $U$ should be swapped with $\emptyset$ and vice versa.
\newline
[examples in word]
\newpage
\subsection{Generalised Set Operations}
Unions and Intersections; A saga:
$$1 \text{. } \cup_{i=1}^{n}A_i = A_1 \cup A_2 \cup \dots \cap A_n,$$
$$2 \text{. } \cap_{i=1}^{n} A_i = A_1 \cap A_2 \cap \dots \cap A_n.$$
Example:
\newline
\newline
$$A_k = {k, k + 1};$$
$$= \cup_{i=1}^{3}A_k = A_1 (\{1, 2\}) \cup A_2 (\{2, 3\}) \cup A_3 (\{3, 4\}),$$
$$= \{1, 2, 3, 4\}.$$
[example in word]
\newpage
\subsection{Russel's Paradox}
A set may contain another set as one of its elements.
\newline
This raises the possibility that a set may contain itself as an element.
\newline
\newline
\textbf{Problem: Try to let S be the set of all sets that are not elements of themselves, i.e.,}
$\mathbf{S = \{ A | A }\textbf{ is a set and } \mathbf{A \not\in A \}}.$
\newline
\textbf{Is S an element of itself?}
\newline
i)	If $S \in S$, then the definition of S implies that $S \not \in S$, a contradiction.
\newline
ii)	If $S \not \in S$, then the definition of S implies that $S \in S$, also a contradiction.
Hence neither $S \in S$ nor $S \not \in S$. This is Russell’s paradox.
\newpage
\subsection{Cartesian Product}
[example in word]
\newline
The Cartesian product of two sets A and B, denoted by $A \times B$, is the set of all ordered pairs from A to B:
\newline
\newline
\boxed{A \times B = \{(a,b)|a \in A \text{ and } b \in B\}}
\newline
\newline
If $|A| = m$ and $|B| = n$, then we have $|A \times B| = mn$.
\newline
\newline
Sets with more than 2 elements:
\newline
\newline
\textbf{Example:} $A = \{ a, b \}, B = \{ 1, 2, 3 \}.$
$$\text{Cartesian Product }(A \times B) = \{(a, 1), (a, 2), (a, 3), (b, 1), (b, 2), (b, 3)\}$$
(all of the ordered pairs -- combinations)
\newline
\newline
[example in word]
\newline
\newline
When X and Y are small finite sets, we can use an arrow diagram to represent a subset S of $X \times Y$ : we list the elements of X and the elements of Y , and then we draw an arrow from x to y for each pair $(x,y) \in S$.
\newpage
\subsection{Functions}
Example: Take 2 sets $X$ and $Y$, for which we have to find a function.
$$X = \{\text{all MATH 1081 students} \}, Y = \{ 0, 1, \dots , 84, 85, \dots, 100\}.$$
X: number of students; Y: marks from $0 - 100$.
\newline
Take function $f: X \to Y; \text{where } X \text{ is the domain and } Y \text{ is the co domain}.$
\newline
Ie, $f(x)$ = X's mark (Y).
\newline
Function $f: X \to Y$ satisfies $\{ (x, f(x)) | x \in X\} \subseteq X \times Y \text{ so that, for each } x \in X$;
\newline
1. $f(x)$ exists
\newline
2. $f(x)$ is unique
\newline
\newline
[example in word]
\newline
\newline
\boxed{\text{Note: be vary of the one-to-one function property lol}}
\newline
\newline
\underline{ Floor function and ceiling functions:}
\newline
\newline
1. Floor function (rounds down; smallest integer):
\newline
\newline
\boxed{\lfloor x \rfloor = \text{ max } \{ z \in Z | z \leq x\}.}
\newline
\newline
2. Ceiling function (rounds up; largest integer):
\newline
\newline
\boxed{\lceil x \rceil = \text{ min } \{ z \in Z | z \geq x\}.}
\newline
\newline
[example in word]
Domain/codomain: $\lfloor x \rfloor / \lceil x \rceil: \mathbb{R} \to \mathbb{Z}.$
\newline
\newline
Range($\lceil x \rceil$) = $\mathbb{Z}.$
\newline
\newline
[example in word]
\newpage
\subsection{Image and Inverse Image}
- The image of a set $A \subseteq X$ under a function $f : X \rightarrow Y \text{ is } f(A) = \{ y \in Y \text{ | } y = f(x) \text{ for some } x \in A\} = \{f(x)|x \in A\}$.
\newline
\newline
- The inverse image of a set $B \subseteq Y$ under a function $f : X \rightarrow Y is f^{-1}(B) = \{ x \in X |f(x) \in B \}$.
\newline
\newline
(image is just function values in the doman and inverse image is function values in range).
\newline
\newline
\boxed{\text{note: this is just function and inverse functions.}}
\newline
\newline
[example in word]
\newpage
\subsection{Injective, Surjective, Bijective}
Formal Definitions:
\newline
\newline
Recall that if f is a function from X to Y , then   for every $x \in X$, there is exactly one $y \in Y$ such that $f(x) = y$.
\newline
\newline
  1. We say that a function $f : X \rightarrow Y$ is injective or one-to-one if, for every $y \in Y$ , there is at most one $x \in X$ such that $f(x) = y$. 
  \newline
  Example: for all $x_1,x_2 \in X$, if $f(x_1) = f(x_2)$ then $x_1 = x_2$.  
  \newline
  \newline
  2. We say that a function $f : X \rightarrow Y$ is surjective or onto if, for every $y \in Y$ , there is at least one $x \in X$ such that $f(x) = y$.
  the range of f is the same as the codomain of f (range(f) = Y).
  \newline
  \newline
  3. We say that a function $f : X \rightarrow Y$ is bijective if $f$ is both injective and surjective (one-to-one and onto).   
  \newline
  \newline
  for every $y \in Y$ , there is exactly one $x \in X$ such that $f(x) = y$.
  \newline
  [example in word]
  \newpage
  \subsection{Composition of Functions}
    For functions $f : X \rightarrow Y$ and $g : Y \rightarrow Z$, the composite of f and g is the function $g \circ f : X \rightarrow Z$ defined by $(g \circ f)(x) = g(f(x))$ for all $x \in X$.
    \newline
    \newline
    The composite function $g \circ f$ exists whenever the range of f is a subset of the domain of g.
    \newline
    \newline
    In general, $g \circ f$ and $f \circ g$ are not the same composite functions. Associativity of composition (assuming they exist): $h \circ (g \circ f) = (h \circ g) \circ f$.  
\newline
\newline
\textbf{ Example: Take sets $X = \{\text{ all MATH1081 students }\}, Y = \{ 0, 1, \dots, 100 \}, Z = \{F, P, CR, D, HD\}.$  }
\newline
\newline 
Maps: $f = X \rightarrow Y; g = Y \rightarrow Z.$
\newline
\newline
A) $g \circ f: X \rightarrow Z.$
\newline
$(f \circ g) (y) = f(g(y)).$
\newline
[examples in word]
\newpage
\subsection{Identity and Inverse Functions}
\underline{ Identity Function:}
\newline
\newline
\boxed{i_x: x \rightarrow x; i_x(x) = x.}
\newline
\newline
For any function $f : X \rightarrow Y$ , we have $f \circ i_x = f = i_y \circ f$.   A function $g : Y \rightarrow X$ is an inverse of $f : X \rightarrow Y$ if
$g(f(x)) = x$ for all $x \in X$ and $f(g(y)) = y$ for all $y \in Y$,
\newline
or equivalently, $g \circ f = i_x$ and $f \circ g = i_y$ .
\newline
\newline
\boxed{\text{1. A function can have at most one inverse.}}
\newline
\newline
If $f : X \rightarrow Y$ has an inverse, then we say that $f$ is invertible, and we denote the inverse off by $f^{-1}$. Thus,$f^{-1} \circ f = i_x$ and $f \circ f^{-1} = i_y$.
\newline
\newline
If $g$ is the inverse of $f$, then $f$ is the inverse of $g$. Thus, $(f^{-1})^{-1} = f$.
\newline
\newline
[example in word]
\newpage
\subsection{Inverse Function Proofs}
\underline{ Theorem and Proof: }
\newline
\newline
\boxed{1. \text{ A function } f: X \rightarrow Y \text{ has at most 1 inverse} }
\newline
\newline
Proof:
\newline
\newline
$$\text{ Let } g_1, g_2: Y \rightarrow X \text{ be inverse of }f.$$
$$\text{ Then }: g_1 = g_i \circ i_y$$
$$ = g_i \circ (f \circ g_2)$$
$$ = (g_i \circ f) \circ g_2$$
$$ = i_x \circ g_2$$
$$ = g_2 \text{ End of proof }.$$
[example in word]
\newpage
\section{Number Theory and Relations}
\subsection{Numbers and Divisibility}
[topic 2 done in word (SteelsSlides1): maybe put in definitions here ?? that depends]
\newline
\newline
Number Set Notation:
\newline
\newline
1. The positive integers: $\mathbb{Z}^+ = \{ 1, 2, 3, \dots\},$
\newline
2. The natural numbers: $\mathbb{N} = \{ 0, 1, 2, 3, \dots\},$
\newline
3. The integers: $\mathbb{Z} = \{\dots, -2, -1, 0, 1, 2, \dots\},$
\newline
4. The rational numbers: $\mathbb{Q} = \{ m/n : m \in \mathbb{Z}, n \in \mathbb{Z}^+ \},$
\newline
5. The real numbers, $\mathbb{R}$ and the complex numbers $\mathbb{C}.$
\newline
\newline
Tests (Divisibility):
\newline
\newline
1. $2\text{ | } N$ if and only if the decimal expansion of N ends in an even integer
\newline
2. $5 \text{ | } N$ if and only if the last decimal digit of N is 5 or 0.
\newline
3. $3 \text{ | } N$ if and only if the sum of the decimal digits of N is divisible by 3.
\newline
3': $9 \text{ | } N$ if and only if the sum of the decimal digits of N is divisible by 9.
\newline
4. $11 \text{ | } N$if the alternating sum of the decimal digits of N is divisible by 11. 
\newline
(example: $1232 = 1 - 2 + 3 - 2 = 0$)
\newline
\newline
[proof in word]
\newpage
\subsection{Primes}
[in word]
\newline
\newline
\underline{Primes Definition: Formal:}
Another way of saying this is if $p$ is prime:
$$x \text{ | } p	\text{ implies }	x \in \{ -1,1,-p,p \}$$.
\newline
\newline
\underline{Theorems:}
\newline
\newline
1. If p is prime and $p | ab$, then $p | a$ or $p | b$,
\newline
2. If n is composite, then it has a prime factor less than or equal to $\sqrt[2]{n}$,
\newline
3. If no prime less than or equal to $\sqrt[2]{n}$ divides $n$ then $n$ is a prime,
\newline
4. Every integer $n \geq 2$ can be written uniquely as a product of a finite number of primes in increasing order i.e. $n = p_1^{m_1} * p_2^{m_2} \dots p_k^{m_k}$
for primes $p_1 < p_2 < \dots < p_k$ and exponents $m_1,m_2, \dots ,m_k \in \mathbb{Z}^+$.
\newline
\newline
Open Results about Primes:
\newline
\newline
1. A prime of the form $2^n + 1$ is called a Fermat prime.
\newline
2. A prime of the form $2^{n} - 1$ is called a Mersenne prime.
\newline
3. Two primes that differ by 2, are called twin primes.
For example, $3$ and $5$ are twin primes; so are $29$ and $31$.
\newline
4. The Goldbach Conjecture is that they are: it has been proved true for all numbers with fewer than about $17$ digits.
\newpage
\subsection{Common Divisors and Multiples}
[mostly on word]
\newline
All $a,b \in \mathbb{Z}$ have (at least) one common divisor, namely $1$, and so we can define the following:
\newline
For $a,b \in \mathbb{Z}$, not both zero, the positive integer d such that
$$1.	d \text{ | } a and d \text{ | } b,$$
$$2.	If c \text{ | } a and c \text{ | } b then c \le d.$$
is called the greatest common divisor of $a$ and $b$.
We write $d = gcd(a,b)$.
\newline
\newline
\boxed{ \text{ Begin by writing $a$ and $b$ as a product of primes. }}
\newline
\newline
\underline{ Properties of GCD:}
\newline
\newline
1. $\text{gcd }(a,b)$ is not affected by the signs of $a$ or $b$
\newline
2. Condition $(2)$ in the definition of gcd can be replaced by $(2')$ if $c \text{ | } a$ and $c \text{ | } b$ then $c \text{ | } d$.
\newline
3. For $a \in \mathbb{Z}^+, \text{gcd }(a,0) = a$.
\newline
\newline
\underline{Least Common Multiple}
\newline
\newline
All $a,b \in \mathbb{Z}$ have (at least) one common multiple, namely $ab$, and so we can define the following:
For $a,b \in \mathbb{Z}$, not both zero, the positive integer l such that
$$1)	a \text{ | } l \text{ and } b \text{ | } l$$
$$2)	\text{If } a \text{ | } c \text{ and } b \text{ | } c \text{ then } l \le c \text{ is called the least common multiple of } a \text{ and } b.$$ 
\newline
We write $l = \text{ lcm }(a,b)$.
\newline
\newline
Theorem:
\newline
\newline
\boxed{\text{For all positive integers $a$ and $b$; gcd}(a,b) x \text{ lcm }(a,b) = ab.}
\newpage
\subsection*{Quotient and Remainder}
\
[mostly in word]
\newline
\newline
\underline{The Quotient-Remainder Theorem (aka The Division Algorithm)}
\newline
\newline
If $a \in \mathbb{Z}$ and $b \in \mathbb{Z}^+$, then there exist unique $q,r \in Z$ such that (q: quotient; r: remainder):
$$a = bq + r	\text{ and }	0 \le r < b.$$
\newline
\newline
\boxed{\text{ Note: $q$ can be found using floor function; $q = \lfloor a / b \rfloor.$; then $r = a - qb$.}}
\newpage
\subsection{The Euclidean Algorithm}
\
[mostly in word]
\newline
\newline
\boxed{\text{If $a = bq + r$ then gcd$(a,b) =$ gcd$(b,r)$.}}
\newline
\newline
\underline{The Euclidian Algorithm: General Case [steps]}
\newline
\newline
$1)$	Let $a$ and $b$ be integers with $a > b \geq 0.$
\newline
$2)$	If $b = 0$, then gcd$(a,b) = a.$
\newline
$3)$	If $b > 0$, use the Quotient-Remainder theorem to write $a = bq + r$ where $0 \leq r < b$.	Then by our prevous result, gcd$(a,b) =$ gcd$(b,r).$
\newline
$4)$	Repeat steps 2 and 3 to find gcd$(b,r)$.
\newline
\newline
Example: Find gcd$(708, 540)$
$$708 = 540 \cdot 1 + 168,$$
$$540 = 168 \cdot 3 + 36,$$
$$168 = 36 \cdot 4 + 24,$$
$$36 = 24 \cdot 1 + 12,$$ 
$$24 = 12 \cdot 2 + 0.$$
So,
$$\text{gcd }(708,540) = 12.$$
\boxed{\text{Note: gcd is the last non-zero remainder.}}
\newline
\newline
\underline{Bezout's Identity}
\newline
\newline
For $a,b \in Z$ not both zero, there exist integers $x$ and $y$ (not unique) such that:
\newline
\newline
\boxed{ \text{gcd }(a,b) = ax + by.}
\newline
\newline
\boxed{\text{Theorem: Integers $a$ and $b$ are relatively prime if and only if there exists $x,y \in Z$ such that $ax + by = 1$.}}
\newline
\newline
\underline{Extended Euclidean Theorem:}
The Extended Euclidean Algorithm is a more efficient way of finding the numbers in Bézout's Identity: In looking for gcd$(a,b)$, assume $a > b > 0$.
\newline
1. We make up a table with five columns labelled $i, q_i, r_i, x_i, y_i$, where $i$ labels the rows.
\newline
2. We set row $1$ to be $1,0,a,1,0$ and row $2$ to be $2,0,b,0,1.$
Thus $q_1 = q_2 = 0; r_1 = a, r_2 = b; x_1 = y_2 = 1; x_2 = y_1 = 0.$
\newline
3. Then for $i$ from $3$ onwards, $q_i$ is the quotient on dividing $r_{i-2}$ by $r_{i-1}$ ($a$ divided by $b$ in the first case).
\newline
4. Then subtract $q_i$ times the rest of row $i - 1$ from row $i - 2$.
\newline
5. Repeat until we get $r_{n+1} = 0$ for some $n$, then stop.
Then the gcd is $r_n$ and $r_n = ax_n + by_n$, that is the last row before $r_i$ was zero gives the gcd, the $x$ and the $y$.
\newline
In fact a similar identity holds at each step: $r_i = ax_i + by_i$.
\newpage
\subsection{Modular Arithmetic}
[mostly in word]
\newline
\newline
Let $m \geq 2$ be an integer. We say that a and b are congruent modulo $m$ if $m | (a - b)$.
\newline
We write this as:
$$a \cong b (mod m).$$
The reason we have taken our modulus m to be greater than $2$ is that
\newline
1)	As $m | (a - b)$ iff $- m | (a - b)$, there is nothing to be gained from using negative moduli.
\newline
2)	All numbers are congruent modulo $1$, so that is not interesting.
\newline
3)	divisibility by $0$ is not defined.
\newline
\newline
\underline{Theorem}
\newline
For integers $a, b$and $m, a \cong b (mod m)$ if and only if there is an integer $k$ such that $a = b + km$.
\newline
\newline
\underline{Arithmetic with Congrences}
\newline
Suppose $a \cong b (mod m)$ and $c \cong d (mod m)$.
\newline
Then
$$(1a) (a + c) \cong (b + d) (mod m).$$
$$(1b) (a - c) \cong (b - d) (mod m).$$
$$(2)	ac \cong bd (mod m).$$
$$(3)	an \cong bn (mod m) for all n \in N.$$
$$(4)	If k \text{ | } m then a \cong b (mod k).$$
\boxed{note: never divide congruences}
\newline
\newline
Applications of Congruence Arithmetic:
\newline
1. Pseudo-random Numbers
\newline
2. Equations with no solutions
\newpage
\subsection{Congruence Equations}
(note: just simple forms, yeah?)
\newline
\newline
Linear Congrence equation form: $ax \equiv b (\text{ mod }m)$.
\newline
\newline
[example in slide07 pdf]
\newline
\newline
\textbf{Tricks. }
\newline
$\bullet $ If, in $ax \equiv b (\text{ mod }m)$, $\text{gcd}(a, m) > 1$ but it does not divide $a$, then the congruence has no solution.
\newline
$\bullet $ If $gcd(a, m) = 1$, then it has a unique solution for modulo $m$.
\newline
$\bullet $ Otherwise, divide everything by $d = gcd(a, m),$ then find solutions.
\newline
$\bullet $ Once you have a solution; if the equation was divided by $d$; then the formula for general solutinos is $k + q (m/d) < m$ where $k$ is the first solution and $q \in \mathbb{R} - {0}$.
\newpage
\subsection{Relations}
Rule of $f: A \rightarrow B$ is that takes each element $a \in A$ to exactly one of $b \in B$.
\newline
\newline
\boxed{\text{Set $A$: domain; Set $B$: codomain}}.
\newline
\newline
$\bullet $ $f:A \rightarrow B$ is also cartesian product of $A \times B$ which contains one and only one ordered pair of $(a, b)$ for each $a \in A$.
\newline
$\bullet $ One-to-one function: iff there is at most one pair for every $b \in B$.
\newline
$\bullet $ Onto function: iff there is at least one pair for every $b \in B$.
\newline
\newline
\textbf{ Examples of relations between sets. }
\newline
\newline
$\bullet $ Subset;
\newline
$\bullet $ Congruence (between integers or e.g. triangles);
\newline
$\bullet $ Divisibility;
\newline
$\bullet $ Less than;
\newline
$\bullet $ Equality (most important one)!
\newline
\newline
Relation of $A x B$; $R$:
\newline
\newline
$\bullet $ We call $A$ the domain of $R$ and $B$ the codomain of $R$.
\newline
$\bullet $ If $(a, b) \in R$, we say that $a$ is related to $b$ (by $R$), written as $a \space R \space b$.
\newline
$\bullet $ If $(a, b) \not \in R$ we write $a \space \not R \space b$.
\newline
\newline
Relation from set $A$ to itself is called \textbf{ a relation on $A$. }
\newline
\newline
Relation $f: A \rightarrow B$ gives: $\{(a, f(a))\} \subseteq A \times B.$
\newline
\newline
[example in slide 08 pdf]
\newline
\newline
Ternary Relation on $\mathbb{Z}:$ $\mathbb{Z} \times \mathbb{Z} \times \mathbb{Z}.$  (for $A x B$, it would be a binary relation).
\newline
\newline
Can generalise it for 3 sets: $A \times B \times C;$ example would be a modular statement like $a \equiv b (\text{ mod } m).$
\newline
\newline
\textbf{Representing relations on finite sets. } For finite sets $A = \{a_1, a_2, \dots a_m \}$ and $B = \{b_1, b_2, \dots , b_n\}$, describe relation from $A$ to $B$ by:
\newline
\newline
$\bullet $ Arrow diagrams: dots (vertices) labelled $a_i \text{ to } b_j$ and join them by an arrow iff $(a_i, b_j) \in \mathbb{R}.$
\newline
$\bullet $ Rectangular array of numbers (a matrix) $M_R$ where entry in $i$th row and $j$th column is $m_{ij}$ and where: $m_{ij} = 1 \text{ if } (a_i, b_j) \in \mathbb{R}$ and $0$ otherwise.
\newline
\newline
[example in 08 pdf]
\newline
\newline
$\bullet $arrow diagrams are sometimes called directed graphs or digraph.
\newpage
\subsection{Properties of Relations}
\textbf{Reflexive Relation:}
\newline
\newline
\boxed{\text{Reflexive iff $(x, x) \in R$ that is $x \space R \space x$}}
\newline 
\newline
Drawn as:
\newline
$\bullet $ a loop in a vertex in directed graph
\newline
$\bullet $ main diagonal is all 1s
\newline 
\newline
[example in slide 9 pdf]
\newline
\newline
\textbf{Symmetric Relation:}
\newline
\newline
\boxed{\text{Symmetric relation $R$ on set $A$ iff $(x, y) \in A$ that is $x \space R \space y$ implies $y \space R \space x$ }}
\newline 
\newline
Drawn as:
\newline
$\bullet $ arrow pointing both ways (ie $x \leftrightarrow y$) in a directed graph
\newline
$\bullet $ matrix $M_R$ is symmetric about main diagonal. 
\newline 
\newline
[example in slide 9 pdf]
\newline
\newline
\textbf{Antiymmetric Relation:}
\newline
\newline
\boxed{\text{Antiymmetric relation $R$ on set $A$ iff $(x, y) \in A$ if both $x \space R \space y$ and $y \space R \space x$ then $x = y$.}}
\newline 
\newline
Drawn as:
\newline
$\bullet $ arrow pointing both ways (ie $x \leftrightarrow y$) in a directed graph should NEVER appear
\newline
$\bullet $ matrix $M_R$ if $i \not = j$, then either $m_{ij} = 0$ or $m_{ji} = 0$ (or both).
\newline 
\newline
[example in slide 9 pdf]
\newline
\newline
\textbf{Transitive Relation:}
\newline
\newline
\boxed{\text{Transitive relation $R$ on set $A$ iff for all $x, y, z \in A$, if both $x \space R \space y$ and $y \space R \space x$ then $x \space R \space Z.$}}
\newline 
\newline
Drawn as:
\newline
$\bullet $ arrow pointing (ie $x \rightarrow y \rightarrow z; x \rightarrow z$) and (if $x \rightarrow y$ then $x$ and $y$ must be reflexive) in a directed graph
\newline
$\bullet $ matrix $M_R$, each non-zero entry in $M^2_R$ is also a non-zero in $M_R$. 
\newline 
\newline
[example in slide 9 pdf]
\newline
\newline
Transitive closure: adding all relative transitive pairs together. 
\newpage
\subsection{Equivalence Relations}
\textbf{Equivalence Relation. } iff it is reflexive, symmetric and transitive. I.e;
\newline
\newline
$\bullet $ For all $x \in A, x \sim x$;
\newline
$\bullet $ For all $x, y \in A, x \sim y \text{ implies } y \sim x$;
\newline
$\bullet $ For all $x, y, z \in A, x \sim y \text{ and } y \sim z \text{ implies } x \sim z.$
\newline
\newline
($\sim$: equivalent; equivalent under $\sim$; congruence modulo $m$ is an equivalence relation)
\newline
\newline
[example in slide 10 pdf]
\newline
\newline
\textbf{Equivalence class. } Setting $\sim$ be an equivalence relation on set $A$; $a \in A$; equivalence class $[a]$:
\newline
\newline
\boxed{[a] = \{x \in A: x \sim a\}}
\newline
\newline
set of all elements that are equivalent to $a$.
\newline
\newline
[example in slide 10 pdf]
\newline
\newline
\textbf{Theorem: } Set of equivalence classes of $A$ under equivalence relation on $A$ is a partition of $A$;
\newline
$(i) $ $A$ is the union of all the equivalence classes and 
\newline 
$(ii) $ different equivalence classes are disjoint.
\newline
\newline
[proof + examples in slide 10 pdf]
\newline
\newline
\textbf{Natural representative. } finding the simplest member of each equivalence class. (could be smallest); set of natural representatives is \textbf{ quotient set. }
\newline
\newline
[example in slide 10 pdf]
\newpage
\subsection{Partial Orders}
\textbf{Partial order } iff it is reflexive, antisymmetric and transitive. I.e;
\newline
\newline
$\bullet $ For all $x \in A, x \sim x$;
\newline
$\bullet $ For all $x, y \in A, x \sim y \text{ and } y \sim x \text{ implies } x = y$;
\newline
$\bullet $ For all $x, y, z \in A, x \sim y \text{ and } y \sim z \text{ implies } x \sim z.$
\newline
\newline
\boxed{\text{$\preceq$: precedes; used to denote a partial order}}
\newline
\newline
\boxed{\text{For $a, b$ where atleast $a \preceq b$ or $b \preceq a$, it is called total order.}}
\newline
\newline
[example in slide 11 pdf]
\newline
\newline
\textbf{Posets. } Posets (partially ordered sets) is a set $A$ togehter with $\preceq$, denoted by $(A, \preceq)$. For poset $(A, \preceq)$
\newline
$\bullet $ If $a \preceq b$ then $a$ is a predecessor of $b$.
\newline
$\bullet $ If $a \preceq b$ but $a \not = b$ (ie $a \prec b$) then we say $a$ strictly precedes $b$. 
\newline
$\bullet $ We say $a$ is an immediate predecessor of $b$ iff $a \prec b$ and there is no $c \in A$ such that $a \prec c \prec b.$ 
\newline
$\bullet $ Elements $a, b$ are comparable if either $a \preceq b$ or $b \preceq a$. Otherwise, $a, b$ are incomparable.
\newline
\newline
Set with total order is called totally ordered set (total order is partial order when any two elements are comparable).
\newline
\newline
[example in slide 11 pdf]
\newline
\newline
\textbf{Hasse Diagrams. } Small poset with directed graph:
\newline
$\bullet $ Elements of $A$ are drawn as vertices.
\newline
$\bullet $ No loops drawn to indicate $a \preceq a$ (is already assumed).
\newline
$\bullet $ If $a \prec b$ then $a$ is drawn longer than $b$ (assume all arrow point upwards).
\newline
$\bullet $ If $a$ is an immediate predecessor of $b$ then upwards line is drawn from $a$ to $b$.
\newline
$\bullet $ Any line that can be deduced from transitivity is omitted.
\newline
\newline
[example in slide 11 pdf]
\newline
\newline
\textbf{Further definitions. } Let $(A, \preceq)$ be a poset.
\newline 
$\bullet $ Maximal: $a \in A$ if no $b \in A$ with $a \prec b$.
\newline 
$\bullet $ Minimal: $a \in A$ if no $b \in A$ with $b \prec a$
\newline 
$\bullet $ Greatest Element: $a \in A$ iff for all $b \in A, b \preceq a$.
\newline 
$\bullet $ Least Element: $a \in A$ iff for all $b \in A, a \preceq b$.
\newline 
\newline
\textbf{Note. } slight difference between greatest element and maximal element.
\newline 
\newline
[example in slide 11 pdf]
\newline 
\newline
Suppose $(A, \preceq)$ and $S \subseteq A$. Then: 
\newline
$\bullet $ Upper Bound of $S$: $a \in A$ if $s \preceq a \forall s \in S$.
\newline
$\bullet $ Lower Bound of $S$: $a \in A$ if $a \preceq s \forall s \in S$.
\newline
$\bullet $ Least Upper Bound of $S$: $a \in A$ if it is an upper bound for $S$ and for every otehr upper bound $b$ of $S, a \preceq b$.
\newline
$\bullet $ Greatest Lower Bound of $S$: $a \in A$ if it is a lower bound for $S$ and for every otehr lower bound $b$ of $S, b \preceq a$.
\newline 
\newline 
[example in slide 11 pdf]
\newpage
\section{Logic and Proofs}
\subsection{Introduction to Logic and Proofs}
Mathematical proof: consists of logical deduction on the basis of agreed premises. Apart from human error the results are certain.
\newline
\newline
\underline{ Techniques for Proof: }
\newline
\newline
•	Always explain what you are doing, and your reasons for drawing your conclusions.
\newline
•	Simplify!
\newline
•	Keep the aim in mind.
\newline
•	Plan a solution. 
\newline
• Work on one side of an equation or inequation to relate it to the other.
\newline
\newline
To study proofs: you always have to practice!! No matter what. There are techniques,
but most of the time you will have to practice proofs.
\newline
What we study is methods of proof and logic.
\newpage
\subsection{Example of Proofs}
[in word?? mostly?? i think so yeah]
\newline
\newline
Example: $\frac{1}{1000} - \frac{1}{1001} < \frac{1}{1000000}.$
\newline
\underline{Techniques:} 
\newline
1. Common denominator (simple answer); 
\newline
2. Reciprocals; 1/2a $<$ 1/a kinda way (bigger denominator smaller fractions)
\newline
3. Simplify
\newline
\newline
\textbf{Proof}. We have
$$\frac{1}{1000} - \frac{1}{1001} = \frac{1001 - 1000}{1000 \text{ x } 1001} = \frac{1}{1001000}$$
But $1001000 > 1000000$, and both are positive numbers, so
$$\frac{1}{1001000} < \frac{1}{1000000}$$  
Therefore
$$\frac{1}{1000} - \frac{1}{1001} < \frac{1}{1000000}.$$  
\newline
\boxed{\text{It is handy to use calculators, but this is generally bad practice as there is no understanding.}}
\newline
\newline
Equality proofs should ideally not be proved by calculators!
\newpage
\subsection{Further Examples fo Proofs}
Another example: 
\newline
Example: $\sqrt[8]{8!} < \sqrt[9]{9!}$.
\newline
[rough working]
$$=> 8! < \sqrt[9]{9! ^ 8},$$
$$=> (8!)^ 9 < (9!)^ 8, $$
$$=> (1 \times 2 \times \dots \times 8)^9 < (1 \times 2 \times \dots \times 9)^8, $$
$$=> (1 \times 2 \times \dots \times 8)^9 < (1 \times 2 \times \dots \times 8)^8 \times 9^8,$$
$$=> 1 \times 2 \times \dots \times 8 < 9^8.$$
which is obviously less than 9 times 9 times etc, therefore making the statement true.
\newline
\newline
However, you can't start like this. Ie, you have to start with the fact that is true, then ending up with the question.
\newline
You have to check if it can be reversed (ie reversing steps in a proof: is that possible?).
\newline
(in this case, you can reverse it: proof in word.)
\newline
\newline
\textbf{ Things to learn from this proof:}
\newline
\newline
$\bullet$ Go back to definitions (expand the definition)
\newline
$\bullet$ Simplify!
\newline
$\bullet$ A proof is often discovered by working backwards; but it must often be written forwards.
\newline
$\bullet$ Explain logical and technical steps in words (with punctuation!)
\newpage
\subsection{Generalisation and 'All' Statements}
$\bullet$ For examples such as $\frac{1}{1000} - \frac{1}{1001} < \frac{1}{1000^2}$, we can replace $n$ with $1000$ to give:
$$\frac{1}{n} - \frac{1}{n + 1} < \frac{1}{n^2},$$
where $\forall n \in \mathbb{Z}^+; \frac{1}{n} - \frac{1}{n + 1} < \frac{1}{n^2}.$
\newline
\newline
This is called a \textbf{ universal statement }(an 'All' statement). The replacement of $n$ with $1000$ as such is called \textbf{generalisation}.
\newline
[proof for the n statement in word]
\newline
\newline
\underline{Things to learn:}
\newline
\newline
$\bullet$ A common proof pattern for $\forall x \in A, \text{property } B \text{  holds is:}$
$$\text{Let } x \in A.
\dots 
\text{Therefore $x$ has a property  } B.$$
$\bullet$ Cannot be proved by listing examples
\newline
\newline
\underline{All Statement Written in:}
\newline
$\bullet  \forall x \in A, \text{ property $B$ holds};$
\newline
$\bullet \text{ For Each } x \in A, x \in B;$
\newline
$\bullet \text{ every } A \text{ is a } B;$
\newline
$\bullet \text{ if } x \text{ is an } A, \text{then } x \text{ is a } B;$
\newline
$\bullet \text{ No } A \text{ is not a } B;$
\newline
$\bullet \forall x \in A, x \in B;$
\newline
$\bullet$ $B$ is true if $A$ is true;
\newline
$\bullet$ $A$ is true only if $B$ is true;
\newline
$\bullet$ property $A$ is sufficient(information) for property $B$ to hold;
\newline
$\bullet$ property $B$ is necessary for property $A$ to hold;
\newline
\newline
[example in word]
\newpage
\subsection{Exhaustion of Cases}
[proofs in word]
\newline
\newline
\underline{Things to learn from these sort of proofs:}
\newline
$\bullet$ Explain what you are doing
\newline
$\bullet$ Outline proof like how you would outline 'all' statement Proofs
\newline
$\bullet$ Often useful where there is natural division of the problem into smaller cases (like where you can use induction or splitting into cases as such);
\newline
Examples include absolute value proofs, (modular) congruences or divisibility.
\newline
$\bullet$ Clearly state the seperate cases, and be sure all cases really are covered.
\newline
$\bullet$ Patterns for "if $A$ then $B$" is:
$$\text{ Suppose that $A$ is true } \dots \text{ Therefore $B$ is true.}$$
\newline
$\bullet$ Expand the definition
\newline
$\bullet$ Keep aim in mind!!! Find general steps, substitute, reduce where necessary.
\newpage
\subsection{Writing Proofs}
[proofs in word]
$\bullet$ Usually involves two stage:
\newline
1. Discover reasons why statement is true
\newline
2. present these reasons as a coherent, carefully written argument.
\begin{theorem}
  Let n be an integer. If $n$ is even, then $n^2$ is even.
  \newline
    $\bullet$ use a basic idea to prove it (for example, $n = 2k$, so $n^2 = 4k^2)$.
\end{theorem}
$\text{ }$
\newline
\textbf{Things to learn from proof}
\newline
$\bullet$ Written in complete sentences, with correct spelling and grammar!
\newline
$\bullet$ Sentence should begin with a word, not a variable/number.
\newline
$\bullet$ In text, writing equations consequtively with no words between them can make it unclear.
\newline
\newline
\boxed{\text{ Bad Practice: Let } n = 2k, k \in \mathbb{Z}.}
\newline
\newline
\boxed{\text{ Good practice: Let } n = 2k, \text{ where } k \in \mathbb{Z}.}
\newline
\newline
\textbf{Structure of a Proof}
\newline
$\bullet$ Begin with a clear statement.
\newline
$\bullet$ Write the word "Proof"; then begin argument.
\newline
$\bullet$ Any notation used; introduce variables properly.
\newline
$\bullet$ Logic of the proof should be clear.
\newline
$\bullet$ Include a conclusion that indicates end. Can include "QED" but not necessary.  
\newline
\newline
\textbf{Helping the reader}
\newline
$\bullet$ Give reasons for all conclusions. 
\newline
$\bullet$ Helpful to explain technical and algebraic steps. 
\newline
$\bullet$ Need to make comments about variables being integers, etc.
\newline
$\bullet$ Try to the get the level of the argument right.
\newpage
\subsection{Converse; If and Only if}
The \textbf{ converse } of "if $A$ then $B$" is "if $B$ then $A$", and so forth.
\newline
\newline
\boxed{\text{ Converse of a statement may or may not be true. }}
\newline
\newline
\boxed{\text{\textbf{Converse fallacy} refers to try and prove the converse of the statement because it is not true.}}
\newline
\newline
[examples in word]
\newline
\newline
\underline{Things to Learn from the proof}
\newline
\newline
$\bullet$ Disproving an "all" statement just means finding a counterexample;
ie find one example where $A$ is true but $B$ is false.
\newline
\newline
\textbf{If and only if} When a statement "if $A$ then $B$" and converse "if $B$ then $A$" are both true, then we can 
combine it to write "$A$ iff $B$" ($A \leftrightarrow B$).
\newline
\newline
Statement can be rephrased as the following:
\newline
$\bullet$ "if $A$ then $B$", conversely "every $A$ is $B$ and every $B$ is $A$".
\newline
$\bullet$ $A$ is a necessary and sufficient condition for $B$;
\newline
$\bullet A \Longleftrightarrow B.$ 
\newline
\newline
[examples in word]
\newline
\newline
\boxed{\text{Note: "Iff" statements really consists of two statements; therefore proof is of two parts.}}
\newline
\newline
\textbf{Common proof patter for iff statements: }
\newline
"Firstly, let $x$ be an $A$"
\dots
"Therefore, $x$ is a $B$."
"Conversely, let $x$ be a $B$.
\dots
Therefore, $x$ is an $A$.
\newline
\newline
\begin{theorem}
  Let $n \in \mathbb{Z}.$ Then, 
    $6 \arrowvert n$ iff both $2 \arrowvert n$ and $3 \arrowvert n.$
\end{theorem}
(1. suppose $6 | n$, then find $2 | n$ and $3 | n$; then prove converse [converse; gcd$(2, 3) = 1$ therefore $2 \times 3 | n$].)
\newline
\newline
\underline{Things to learn from proof: }
\newline
$\bullet$ Expand definitions!!
\newline
$\bullet$ Keep goal in mind.
\newline
$\bullet$ Make logical subdivision clear by writing "Firstly" and "Conversely"
\newline
$\bullet$ Each part follows the normal format for an "if ... then" proof.
\newline
$\bullet$ "Similarily" saves work for the reader, not the writer. 
\newpage
\subsection{"Some" Statements}
Asserts that there exists something which satisfies a certain condition.
\newline
\newline
\boxed{\text{"There exists something such that ..." means that there is one object or more with the given property.}}
\newline
\newline
For example: "there exists $x \in \mathbb{R}$ such that $x^2 = 2$"
\newline
\newline 
\textbf{Examples of "some" statements}
\newline
\newline
$\bullet$ Some integers are positive.
\newline
$\bullet$ There exists a function $f: \mathbb{R} \rightarrow \mathbb{R}$ which is both odd and even.
\newline
$\bullet$ Some students enjoy doing Maple tests.
\newline
$\bullet \text{  } S \neq \emptyset.$
\newline
$\bullet \text{  } 6 \arrowvert n.$
\newline
$\bullet \text{  } a$ is odd.
\newline
\newline
A "some" statement can be written as:
\newline
\newline
$\bullet$ some $A$ is a $B$;
\newline
$\bullet$ some $A$s are $B$s;
\newline
$\bullet$ there exists an $A$ which is (also) a $B$;
\newline
$\bullet$ for some $x \in A$ we have $x \in B$;
\newline
$\bullet$ property $B$ holds for some $x \in A$;
\newline
$\bullet$ $\exists \text{ } x \in A : x \in B$;
\newline
\newline
[example in word]
\newline
\newline
\underline{Things to learn from proofs:}
\newline
\newline
$\bullet$ Discover the proof by working backwards; but write proof in right order.
\newline
$\bullet$ Real work is "behind the scenes"; requires much more knowledge of algebra and calculus.
\newline
$\bullet$ Proof with $f$ gives a "some" statement which has an "all" statement within.
\newpage
\subsection{"Some" statements confused}
Sometimes it is possible to prove a "some" statement without producing any particular object.
\newline
\newline
\textbf{Example: } Let $f(x) = x^5 + 2x - 2.$ Then $f(x) = 0$ for some $x \in [0, 1].$
\newline
Check end points. The graphs are continuous curves.
\newline
\newline
\textbf{Proof. } Let $f(x) = x^5 + 2x - 2.$  Then, $f(0) = -2$, which is negative, while $f(1) = 1;$ which is positive.
So, the graph of $f$, being a continuous curve, must cross $x$-axis somewhere between $0$ and $1$; and at this crossing point we have $f(x) = 0.$
\newline
(existence proof) therefore, it completes the proof.
\newline
\newline
[other example in word]
\newline
\newline
\underline{Things to learn from proofs:}
\newline
\newline
$\bullet $ Needed background knowledge
\newline
$\bullet $ Tedious to find actual $x$ and $y$ values to prove, so find generic solutions that make proofs easier.
\newline
$\bullet $ "Clearly" does not mean you don't have to check the following statement. (ie you have to check if the most obvious statement in the question is also true.)
\newline 
\newline
\textbf{Existence and uniqueness. } A statement of the form "there exists a unique $x \in S$ such that \dots"
\newline
asserts that there is one and only one object having the given property.
\newline
\newline
Proof for such statements:
\newline
$\bullet $ show $\exists$ an object with given property;
\newline
$\bullet $ show there cannot be two diff objects with property. Common proof patter is as follows:
\newline
\newline
"Suppose that $x_1$ and $x_2$ have the same property."
\newline
\dots
\newline
"Therefore, $x_1 = x_2$."
\newline
\newline
[examples in word: $a = bq + r.$]
\newline
\newline
\underline{Things to learn from proof:}
\newline
$\bullet $ Clearly distinguish the "existence" and "uniqueness" parts of the proof.
\newpage
\subsection{Multiple Quantifiers}
$\bullet $ The words "all" and "some" -> quantifiers. A statement can have more than one quantifier.
\newline
\newline
\textbf{Examples:}
\newline
$\bullet $ For every $x \in \mathbb{Z}, \exists y \in \mathbb{Z}$ such that $y > x$.
\newline
$\bullet $ $\exists y \in \mathbb{Z}$ such that for every $x \in \mathbb{Z}, y > x.$ (not the same as the first statement.)
\newline
$\bullet $ $2^29 - 1$ is composite. (rewrite: $2^29 - 1 = ab$ where $a > 1, b > 1$ for some integers $a, b$.)
\newline
$\bullet $ For any prime there is a larger prime.
\newline
$\bullet $ $(6, -1, 5)$ is a linear combination of $(1, 1, 2)$ and $(1, 2, 3).$
\newline
$\bullet $ There is a function $f: \mathbb{R} \rightarrow \mathbb{R}$ which is equal to its own derivative.
\newline
$\bullet $ Every positive real number has a real square root.
\newline
$\bullet $ For all $A, B, C \subseteq U, A \cup (B \cup C) = (A \cup B) \cup C.$
\newline
$\bullet $ The function $f : X \rightarrow Y$ is onto.
\newline
\newline
\begin{theorem}
  Any composite positive integer has a factor greater than $1$ and less than or equal to its square root.
\end{theorem}
Note: Making the quantifiers explicit, the statement is: 
\newline
\newline
\boxed{\text{"for every composite integer $n \text{ }\exists \text{ } c,$ a factor of $n$ such that $c > 1$ and $c \leq \sqrt{n}$".}}  
\newline
\newline
Let $n$ be composite.
\newline
$$n = ab, a > 1, b > 1.$$
Case $I: a \leq \sqrt(n)$
$$c = a, \text{ has required properties}.$$
Case $II: a > \sqrt{n}$
$$c = b; b = \frac{n}{a} < \frac{n}{\sqrt{n}} = \sqrt{n}.$$
Therefore, $c \arrowvert n, 1 < c \leq \sqrt{n}.$
\newline
\newline
\underline{ Things to learn from proof: }
\newline
$\bullet $ Rewrite the given statement if necessary to make the quantifiers clear.
\newline
$\bullet $ Patter of "all" proof: within this, patter of "existence" proof.
\newline
$\bullet $ Another example of "proof by exhaustion" of cases.
\newline
$\bullet $ Expand the defintion!!
\newpage
\subsection{Multiple quantifiers continued (limits)}
\begin{theorem}
  $\lim_{x \to \infty} \frac{4x^2 + 7x + 19}{2x^2 + 3} = 2.$
\end{theorem}
\textbf{Note: } According to definition of a limit, we must show that for every $\epsilon > 0, \exists$ a real number $M$ such that
$$\text{if } x > M \text{ then } \arrowvert \text{ } \frac{4x^2 + 7x + 19}{2x^2 + 3} - 2 \text{ } \arrowvert < \epsilon.$$
\textbf{Proof: } Let $x > 0.$
\newline
Choose $M = $ max$(1, \frac{10}{\epsilon}).$
\newline
Let $x > M. $ Then
$$x > 1 \text{ and } x > \frac{10}{\epsilon}.$$
so
\newline
$$\arrowvert \text{ } \frac{4x^2 + 7x + 19}{2x^2 + 3} - 2 \text{ } \arrowvert = \arrowvert \text{ } \frac{7x + 13}{2x^2 + 3} \text{ } \arrowvert,$$
$$ = \frac{7x + 13}{2x^2 + 3},$$
$$ < \frac{7x + 13}{2x^2} \text{ (because $x > 1$)},$$
$$ = \frac{10}{x},$$
$$ = \frac{10}{\frac{10}{\epsilon}} \text{ (since $x = \frac{10}{\epsilon}$)},$$
$$ = \epsilon.$$
Therefore, $\lim_{x \to \infty} \frac{4x^2 + 7x + 19}{2x^2 + 3} = 2.$
\newline
\newline
\underline{Things to learn from proof:}
\newline
$\bullet $ Expand definition!!
\newline
$\bullet $ Logical structure of statement to be proved is: "for all \dots there exists \dots such that if \dots then \dots",
and proof follows structure properly. 
\newline
$\bullet $ Working backwards is very important!!
\newpage
\subsection{Change of order of quantifiers}
Two adjacent quantifiers of the same kind can be interchanged. Example:
$$\exists \text{ } \alpha \in \mathbb{R}, \exists \text{ } \beta \in \mathbb{R}, (6, -1, 5) = \alpha(1, 1, 2) + \beta(1, 2, 3).$$
means the same as:
$$\exists \text{ } \beta \in \mathbb{R}, \exists \text{ } \alpha \in \mathbb{R}, (6, -1, 5) = \alpha(1, 1, 2) + \beta(1, 2, 3).$$
as both statements exert two variables being real numbers; satifies equation. 
\newline
\newline
Another example:
$$\forall A \subseteq U , \forall B \subseteq U, A \cap B = B \cap A.$$
and
$$\forall B \subseteq U , \forall A \subseteq U, A \cap B = B \cap A.$$
are the same as it is both subsets of $U$.
\newline
\newline
\boxed{\text{"All" and a "some" quantifier may not be interchanged.}}
\newline
\newline
[example in word]
\newline
\newline
\underline{Things to learn from proof:}
\newline
$\bullet $ Clearly distinguish between what is given and what you have to prove
\newline
$\bullet $ "imaginative" step of the proof: come up with 'numbers' to suit the equation. 
\newline
$\bullet $ Working backwards is important
\newline
$\bullet $ Going back to the definition is a profitable idea.
\newpage
\subsection{"Not" and contradiction}
Negation of statement -> asserts that the statement is false. 
\newline
\newline
\boxed{\text{ Notation: $\sim A$, $\neg A$ or $\sim (A).$}}
\newline
\newline
\textbf{Examples}
\newline
$\bullet $ Negation of $2 + 2 = 5$ is "it is false that $2 + 2 = 5$" or $2 + 2 \not = 5.$
\newline
$\bullet $ Negation of $\frac{1}{n} - \frac{1}{n + 1} < \frac{1}{n^2}$ is $\frac{1}{n} - \frac{1}{n + 1} \geq \frac{1}{n^2}.$
\newline
$\bullet $ However, $A \subset B$ is not $A >= B$! it is $A \not\subseteq B.$
\newline
\newline
[example in word]
\newline
\newline
\underline{Things to learn from proof: }
\newline
$\bullet $ This proof is called \textbf{proof by contradiction } or reductio ad absurdum.
\newline
$\bullet $ Deciding statement true or false: work out consequeunces till u find something that is known to be true or false. 
\newline
\newline
Distinguish formats of \textbf{proof by contradiction }. 
\newline
$1.$ "Suppose $A$ is true. \dots Therefore $B$. But $B$ is false. Therefore $A$ is false. " -> \textbf{VALID} reasoning.
\newline
$2.$  "Suppose $A$ is true. \dots Therefore $B$. But $B$ is true. Therefore $A$ is true. " -> \textbf{INVALID} reasoning (dont use this if using proof by contradiction).
\newline
\newline
[example in word]
\newline
\newline
\underline{Things to learn from proof:}
\newline
$\bullet $ Proof by contradiction begins with assuming negation of statemetn to be proved.
\newline
$\bullet $ Be extremely careful with logic and setting out of a proof by contradiction.
\newline
$\bullet $ Make proof easier to read by introducing suitable notation. 
\newline
$\bullet $ Essentially "sub-proof" by contradiction within the main proof by contradiction.
\newline
$\bullet $ Justifiable use of the word "similarily".
\newpage
\subsection{Contrapositive}
\textbf{Contrapositive} of a statement "if $A$ then $B$" is "if not $B$ then not $A$".
\newline
\newline
$\bullet $ Another example: "every $A$ is a $B$" is "anything which is not a $B$ is not an $A$".
\newline
\newline
Logically equivalent to the original. 
\newline
\newline
Possible proof pattern:
\newline
" Suppose $B$ is false \dots then $A$ is false. Therefore, by Contrapositive, $A$ and $B$ would be true."
\newline
\newline
\textbf{Examples}
\newline
$\bullet $ Let $f: X \rightarrow Y.$. Contrapositive of "if $x_1 \not = x_2,$ then $f(x_1) \not = f(x+2)$" is "if $f(x_1) = f(x_2)$ then $x_1 = x_2$."
\newline
$\bullet $ Contrapositive of "if $n^2$ is ever, $n$ is even" is "if $n$ is odd, $n^2$ is odd".
\newline
[other examples in word]
\newline
\newline
\underline{Things to learn from proof: }
\newline
$\bullet $ Contrapositive is valuable and gives more direct information than trying to prove the statement itself.
\newline
$\bullet $ If you're using contrapositive, it is a good idea to say so in the beginning.
\newline
$\bullet $ substitute "if \dots then" proof.
\newline
\newline
[other example in word: $\sqrt{2}$ is rational]
\newline
\newline
\underline{Things to learn from the proof: }
\newline
$\bullet $ As for reductio ad absurdum, proof begins by assuming given statement is false.
\newline
$\bullet $ Proving a number is irrational nearly always involve proof by contradiction.
\newpage
\subsection{More examples of negation and negation of}
More examples of negation:
\newline
\newline
$\bullet $ The negation of "all first-year maths is easy" is "not all first-year maths is easy".
\newline
$\bullet $ Negation of "there exists $x \in \mathbb{R} \text{ such that }x^2 = -1.$" is "there does exists $x \in \mathbb{R} \text{ such that }x^2 = -1.$" or "for all $x \in \mathbb{R}, x^2 \not = -1$".
\newline
$\bullet $ Negation of "for all $x \in U$" is "for some $x \in U, \text{ not } (\dots)$". 
\newline
$\bullet $ Negation of "for some $x \in U$" is "for all $x \in U$, not $( \dots )$".
\newline
\newline
[write a counterexample to disprove all statements and for "some" statements, can prove "all" statement or proof by contradiction.]
\newline
\newline
[example in word]
\newline
\newline
\underline{Things to learn from this proof:}
\newline
\newline
$\bullet $ Try a few examples if the proof is not so obvious. This either proves a counterexample or gives a pattern that you can work with later.
\newline
\newline
\textbf{Negation of multiple quantifiers } The above equivalences can be used repeatedly to simplify the negation of statement containing multiple quantifiers. 
\newline
\newline
\textbf{Example } Simplify the negation of:
$$\forall x \forall y \exists z \dots.$$
\textbf{Solution } The negation is:
$$\sim (\forall x \forall y \exists z \dots ),$$
that is,
$$\exists x \sim(\forall y \exists z \dots),$$
or
$$\exists x \exists y \sim(\exists z \dots),$$
or finally,
$$\exists x \exists y \forall z \sim(\dots).$$
\newline
\newline
[another example (calc) in word]
\newline
\newline
\underline{Things learned from the proof:}
\newline
$\bullet $ Draw pictures to assist intuition.
\newline
$\bullet $ Negation of "if $A$ then $B$" can be expressed as "$A$ and not $B$".
\newline
$\bullet $ First three lines of the proof deal (in correct order) with the three quantifiers in the statement to be proved.
\newpage
\subsection{Mathematical induction 1}
Mathematical induction is useful for proving "all" statements, specially about natural numbers.
\newline
\newline
Mathematical induction of the statement "for all $n \in \mathbb{N}$ consists of two parts:
\newline
$1) $ prove \dots for $n = 0$;
\newline
$2) $ prove that if \dots is true for some value $k \in \mathbb{N}$ then \dots is true for $k + 1$.
\newline
\newline
Likewise, "for all $n \geq n_0, \dots$" can be proved by mathematical induction by:
\newline
$1) $ prove \dots for $n = n_0$:
\newline
$2) $ prove that if \dots is true for some particular value of $n \geq n_0,$ then it is true for $n + 1$.
\newline
\newline
Step $1)$ is called the \textbf{ basis } of proof; Step $2) $ is called the \textbf{ inductive step}.
\newline
\newline
[proof in word]
\newline
\newline
\underline{Things to learn from proof:}
\newline
$\bullet $ Inductive proof format:
\newline
"Let $n = 1$ \dots Therefore result is true for $n = 1$. Now assume that the result is true for some particular $n$. We must prove that \dots Therefore result is true for $n + 1$. By induction, the result is true for all $n \geq 1$."
\newline 
$\bullet $ To make induction work, we need some simple relation between $n$ and $n + 1$.
\newline
\newline
[proof in word]
\newline
\newline
\underline{Things learnt from proof: }
\newline
$\bullet $ Never multiply an inequality by anything unless you are sure that the "anything" (number which is multiplied in this case) is positive or negative.
\newpage
\subsection{Mathematical Induction 2}
\textbf{Strong induction or extended induction}
\newline
\newline
Suppose statement true for $n = 0$ and $n = 1$; and if is true for consecutive integers $n$ and $n + 1$, then it is true for $n \geq 0$.
\newline
\newline
[example in word]
\newline
\newline
\underline{Things to learn from proof:}
\newline
$\bullet $ Here, proof depends upon knowing if it holds true for the previous two values. That is why, always prove for the first two values when this is the case.
\newline
$\bullet $ Keep aim in mind!
\newline
$\bullet $ Usually used for recurrence relations.
\newline
$\bullet $ In case like this, be absolutely clear on what is given and what is required to prove.
\newline 
Extend the method even further; suppose:
\newline
$(i) $ a certain statement is true for $n = 1$;
\newline
$(ii) $ for any particular $n \geq 1$, if the statement is true for $1, 2, 3, \dots n$ then it is true for $n + 1$.
\newline
Then, the statement is true for all $n \geq 1$.
\newline
\newline
[example in word]
\newpage
\subsection{Logic}
\textbf{Logic } is the study of how the truth or falsity of a given statement follows (or not!) from the truth of other statements. For example, given the statements
\newline
\newline
“if G is an Eulerian graph, then no vertex of G has odd degree” and “G is an Eulerian graph”,
\newline
\newline
we may conclude with “no vertex of G has odd degree”.
\newline
\newline
\textbf{Note. } The logic we have used here says nothing about whether the first two statements are actually true or not, but only that if they are true, then the third is also.
\newline
\newline
\textbf{Definition. } A proposition is a statement which is unambiguosly true or false.
\newline
\newline
\textbf{Examples. }
\newline
\newline
$\bullet $ $1 + 1 = 2$;
\newline
$\bullet $ $2 + 2 = 3$;
\newline
$\bullet $ my birthday is on $29$ February;
\newline
$\bullet $ there exist infinitely many primes $p$ for which $2^p - 1$ is also prime. (mersenne primes)
\newline
\newline
Not proposition example: $2 + 2, $ etc.
\newline
\newline
Complicated expressions use logical operators such as:
\newline
$\bullet $ "not": $\sim$;
\newline
$\bullet $ "and": $\land$;
\newline
$\bullet $ "or": $\vee$;
\newline
$\bullet $ "exclusive or": $\oplus$;
\newline
$\bullet $ "if \dots then": $\rightarrow$;
\newline
$\bullet $ "if and only if": $\leftrightarrow$.
\newline
\newline
\textbf{Examples. } if $p, q, r$ are propositions:
\newline
\newline
$\bullet $ $p \land (\sim q)$ means "$p$ and not $q$";
\newline
$\bullet $ $(\sim q) \rightarrow (\sim p)$ is the contrapositive of $p \rightarrow q$;
\newline
$\bullet $ $q \rightarrow p$ is the converse of $p \rightarrow q.$
\newline
These are called "propositional forms".
\newline
\newline
[example in $3b$ pdf]
\newpage
\subsection{Truth Tables}
Logic of propositions: give \textbf{ truth } values $T$ or $F$ for propositions $p, q, r, \dots$ and determine the truth value for a certain commpound composition.
\newline
\newline
[truth value tables in 3b pdf]
\newline
\newline
$\bullet $"And": conjunction; 
\newline
$\bullet $"Or": disjunction (inclusive meaning of or). 
\newline
\newline
[example in 3b pdf]
\newline
\newline
\textbf{Definition. } Two proposition forms are logically equivalent if they have the same truth values for each possible allocation of truth values to variables in them.
\newline
Denote logical equivalence of $P$ and $Q$ by writing $P \Longleftrightarrow Q.$
\newline
\newline
\textbf{De Morgan's Law. } $\sim(p \vee q) \Longleftrightarrow (\sim p) \land (\sim q).$
\newline
\newline
[example in word]
\newline
\newline
If RHS and LHS for particular statements have same truth values in all cases, then $p \land (q \vee r) \Longleftrightarrow (p \land q) \vee (p \land r).$ which is \textbf{ Distributive Law }.
\newline
\newline
\textbf{Defition. } 
\newline
$\bullet $ Propositional form that is always true is called "tautology". 
\newline
$\bullet $ Proposition form that is always false is called "contradiction".  
\newline
$\bullet $ Proposition that is neither a tautology or a contradiction is called "contigency".
\newline
\newline
\textbf{Example }
\newline
$\bullet $ $p \vee (\sim p)$ is tautology.
\newline
$\bullet $ $p \land (\sim p)$ is contradiction.
\newline
$\bullet $ $p \land q$ is a contigency.
\newline
\newline
\textbf{Note. }
\newline
\newline
$1) $ Any two tautologies/contradictions are logically equivalent, as they have same truth value for all cases.
\newline
$2) $ Conversely, if a proposition is equivalent to a tautology, then the proposition itself is a tautology.
\newpage
\subsection{Laws of logical equivalence}
\textbf{Laws of logical equivalence. } let $p, q, r$ be propositional variables, let $T$ be tautology and $F$ a contradiction. Then:
$$\sim (p \vee q) \Longleftrightarrow (\sim p) \land (\sim q),$$
$$p \land (q \vee r) \Longleftrightarrow (p \land q) \vee (p \land r),$$
$$p \vee (\sim p) \Longleftrightarrow T, (1)$$
$$p \land (\sim p) \Longleftrightarrow F, (2)$$
$$\sim(p \vee q) \Longleftrightarrow (\sim p) \land (\sim q) \text{ and } \overline{ A \cup B } = \bar{ A } \cap \bar{ B },$$
$$p \vee (\sim p) \Longleftrightarrow T \text{ and } A \cup \bar{A} = \cal U,$$
$$p \land (\sim p) \Longleftrightarrow F \text { and } A \cap \bar{A} = \emptyset.$$
\textbf{Note. } $(1)$ and $(2)$ are dual pairs of logical equivalences.
\newline
\newline
\textbf{Laws. (most useful logical equivalences)}
\newline
$\bullet $ Commutative Law:
$$p \land q \Longleftrightarrow q \land p \text{ and } p \vee q \Longleftrightarrow q \vee p,$$
\newline
$\bullet $ Associative Law:
$$(p \land q) \land r \Longleftrightarrow p \land (q \land r) \text{ and } (p \vee q) \vee r \Longleftrightarrow p \vee (q \vee r),$$
\newline
$\bullet $ Distributive Law:
$$p \land (q \vee r) \Longleftrightarrow (p \land q) \vee (q \land r) \text{ and } p \vee (q \land r) \Longleftrightarrow (p \vee q) \land (q \vee r),$$
\newline
$\bullet $ Identity laws:
$$p \land T \Longleftrightarrow p \text{ and } p \vee F \Longleftrightarrow p,$$
\newline
$\bullet $ Laws of negation:
$$p \vee (\sim p) \Longleftrightarrow T \text{ and } p \land (\sim p) \Longleftrightarrow F,$$
\newline
$\bullet $ Double negation law:
$$\sim(\sim p) \Longleftrightarrow p,$$
\newline
$\bullet $ Idempotent Laws:
$$p \land p \Longleftrightarrow p \text{ and } p \vee p \Longleftrightarrow p,$$
\newline
$\bullet $ Domination Law:
$$p \vee T \Longleftrightarrow T \text{ and } p \land F \Longleftrightarrow F,$$
\newline
$\bullet $ De Morgan's Law:
$$\sim(p \land q) \Longleftrightarrow (\sim p) \vee (\sim q) \text{ and } \sim(p \vee q) \Longleftrightarrow (\sim p) \land (\sim q).$$
\newline
\newline
[example in pdf 3b]
\newline
\newline
\textbf{If \dots then.} $p \rightarrow q.$ or "$p$ implies $q$" or "if $p$ then $q$". (always true except when $p$ is true and $q$ is false).
\newline
\newline
[truth table for logical equivalence involving $\rightarrow$ in pdf 3b + example]
\newline
\newline
\textbf{Note. } Non-equivalence of two propositional forms would normally be very difficult to prove by "algebraic" methods.
\newline
\newline
\textbf{If and only if. } Biconditional proposition $p \longleftrightarrow q$ is true when $p$ and $q$ are both true or both false, and is false otherwise (like when one is true and one is false, $p \longleftrightarrow q$ is false).
$$p \longleftrightarrow q \Longleftrightarrow (p \rightarrow q) \land (q \rightarrow p).$$
\newline
\newline
$\bullet $ $p \Longleftrightarrow q$ tells something about the two propositional forms;
\newline
$\bullet $ $p \longleftrightarrow q$ is a combination of the two propositional forms into a single form.\
\newline
\newline
[theorem in 3b pdf + proof (for logical equivalence of two forms if and only if the forms in $\longleftrightarrow$ is a tautology.)]
\newline
\newline
\textbf{Definition. } Let $P$, $Q$ be propositional forms. If $P$ is true, $Q$ is true. Then we can say $P$ logically implies $Q$ leading to $P \Longrightarrow Q.$
\newline
\newline
[example in 3b pdf]
\newpage
\subsection{Valid and Invalid arguments}
Argument: “If G is an Eulerian graph, then no vertex of G has odd degree.
G is an Eulerian graph. Therefore, no vertex of G has odd degree.” is made using the rule of inference. 
\newline
$p \rightarrow q$
\newline
$p$
\newline
$\therefore \bar{q}.$
\newline
(q is just the result. yeah)
\newline
\newline
Such an argument is always valid because the conclusion must always be true, provided hypotheses(earlier statements) are true.
\newline
\newline
This rule of inference is aclled "modus ponens".
\newline
\newline
Invalid inference example:
\newline
$p \rightarrow q$
\newline
$p$
\newline
$\therefore \overline{p}.$
\newline
\newline
Since some cases can be true and some can be false.
\newline
\newline
\textbf{Note. } Conclusion of a valid argument need not always be true. (for a true conclusion, we need a valid argument and true hypotheses).
\newline
\newline
Modus Tollens:
\newline
$p \rightarrow q$
\newline
$\sim q$
\newline
$\therefore \overline{\sim p}.$
\newline
\newline
[truth tables of modus ponens in 3b pdf]
\newline
\newline
\begin{theorem}
  An argument $P_1 P_2 \dots P_n \overline{Q}$ is valid if and only if $(P_1 \land \_2 \land \dots \land P_n) \Longrightarrow Q$ that is, if and only if the proposition $(P_1 \land \_2 \land \dots \land P_n) \longrightarrow Q$ is a tautology.
\end{theorem}
[proof in 3b pdf]
\newline
\newline
\underline{Things to learn from proof: }
\newline 
$\bullet $ Structure of an "if and only if" proof.
\newline
$\bullet $ Proof by division into cases.
\newline
$\bullet $ Proof by contrapositive.
\newline
\newline
[example in word]
\newline
\newline
\begin{theorem}
  The rule of inference: $(\sim p) \longrightarrow F \therefore p.$ is valid.
\end{theorem}
[proof in 3b pdf; formal verification of method by proof of contradiction]
\newline
\newline
[problem in 3b pdf + solution]
\newpage
\section{Combinatorics}
\subsection{Brief Introduction 1}
$\bullet $ basically doing counting, with different techniques and such; 
\newline
\newline
$\bullet $ Formulise all techniques taught and apply it to real world as such.
\newpage
\subsection{Addition and Multiplication Principles}
\textbf{Proposition: Addition Principle }
\newline
\newline
Let $A_1, \dots A_n$ be finite sets so that $A_i \cap A_j = \null$ for $i \not = j$.
\newline
Then, $\text{ | } A_1 \cup \dots \cup A_n \text{ | } = \text{ | } A_1 \text{ | } + \dots + \text{ | } A_n \text{ | }.$
\newline
\newline
\textbf{Proposition: Multiplication Principle: }
\newline
Let $A_1, \dots , A_n$ be finite sets. Then,
\newline
$\text{ | } A_1 \times \dots \times A_n \text{ | } = \text{ | } A \text{ | } \dots \text{ | } A_n \text{ | }.$
\newline
\newline
$\bullet $ Addition Principle: can be used for questions such as 'how many choices in total? -> add to total', etc and
\newline
$\bullet $ Multiplication Principle can be used for questions such as "choose one from the others -> total possible combinations"
\newline
\newline
\textbf{Example: }
A restaurant menu has $7$ mains and $3$ desserts.
\newline
\newline
$a) $ If one has to choose a main or a desert (but not both), how many choices would one have?
\newline
\newline
$A_1 = {\text{ choices of mains }}, A_2 = {\text{ choices of desserts }}.$
\newline
$\text{ | } A_1 \cup A_2 \text{ | } = \text{ | } A_1 \text{ | } + \text{ | } A_2 \text{ | } - \text{ | } A_1 \cap A_2 \text{ | },$
\newline
$\text{ | } A_1 \cup A_2 \text{ | } = \text{ | } A_1 \text{ | } + \text{ | } A_2$ since $\text{ | } A_1 \cap A_2 \text{ | } = \null,$
\newline
$\text{ | } A_1 \cup A_2 \text{ | } = 7 + 3 = 10.$
\newline
\newline
$b) $ If one was to choose one main and dessert, how many choices does one have?
\newline
\newline
$A_1 = {\text{ choices of mains }}, A_2 = {\text{ choices of desserts }}.$
\newline
Asking $\text{ | } A_1 \times A_2 \text{ | } = \text{ | } A_1 \text{ | } x \text{ | } A_2 \text{ | },$
\newline
$\text{ | } A_1 \times A_2 \text{ | } = 7 \times 3 = 21.$
\newpage
\subsection{A remark on the addition principle}
\textbf{Example: }
\newline
Last year, $431$ students enrolled in MATH$1081$ and $562$ students enrolled in MATH$1131$. How many students enrolled in either courses?
\newline
\newline
$A) $ We can't tell! Not enough information about number of students enrolled in both courses.
\newline
If $A_1 = {\text{ students in MATH}1081 }, A_2 = {\text{ students enrolled in MATH}1131},$
\newline
\newline
It is not clear that $A_1 \cap A_2 = \null,$ so the addition principle does not apply.
\newpage
\subsection{Warm-up example words in alphabet}
\textbf{Example: }
\newline 
\textbf{1. How many 3 letter words are there in an alphabet?}
\newline
\newline
$A) $ (Choices for letter 1): $26 \times $ (Choices for letter two): $26 \times$ (Choices for letter three): $26  = 26^3.$
\newline
\newline
\textbf{2. How many 3 letter workds contain exactly one vowel?}
\newline
\newline
$A) $ Choices for letter one times choices for letter two times choices for letter $3$ but with $3$ cases, where: 
\newline
$(i) $ vowel position $1$: $5 \times 21 \times 21.$
\newline
$(ii) $ vowel position $2$: $21 \times 5 \times 21.$
\newline
$(iii) $ vowel position $3$: $21 \times 21 \times 5.$
\newline
Total $ = 5 \times 21 \times 21  + 21 \times 5 \times 21 + 21 \times 21 \times 5,$
\newline
$ = 3 \times 5 \times 21 \times 21.$ (where it could be (number of combinations) times (number of words with one vowel); if all the combinations are the same).
\newline
\newline
\textbf{3. How many 3 letter words come before 'EGG' in alphabetical order?}
\newline
\newline
$A) $ 3 ways to do so:
\newline
\newline
$(i) $ First letter is $A, B, C$ or $D$.
\newline
$(ii) $ First letter is $E$ and second letter is $A, B, C, D, E$ or $F$.
\newline
$(iii) $ First letter $E$, second $G$, and third letter is $A, B, C, D, E$ or $F$.
\newline
\newline
$(i)$ = $4 \times 26 \times 26$
\newline
$(i)$ = $1 \times 6 \times 26$
\newline
$(i)$ = $1 \times 1 \times 6$
\newline
Total $ = (4 \times  26 \times 26) + (1 \times 6 \times 26) + (1 \times 1 \times 6),$
\newpage
\subsection{Ordered selection with repetition}
\textbf{Example: }
\textbf{1. How many 3 letter words are there in an alphabet?}
\newline
\newline
A three letter word can be thought of as a function: $w:  \{ 1, 2, 3 \} \rightarrow \{ A, B, C, \dots, Z\}.$ where $w(i) = $ letter in $i^{th}$ position.
\newline
\newline
\begin{theorem}
  Ordered Selection with repetition: Given two finite sets $N$ and $M$ whose
  coordinates are: $\text{ | } N \text{ | } = n$ and $\text{ | } M \text{ | } = m,$
  \newline
  Number of functions $f: N \rightarrow M$ is given by: $\text{Fun }(N, M) = m^n$.
\end{theorem}
\textbf{Note. } secretly think of $N$ as the set of positions of selection $\{ 1, 2, 3, \dots, n\}.$
\newline
\newline
\textbf{Proof. } Let elements of $N$ be called $N = \{ 1, 2, 3, \dots, n \}.$
\newline
\newline
Given $m := (m_1, \dots m_n) \in M \times M \times \dots \times M.$
\newline
Define $f_{\underline{m}}: N \rightarrow M$ by $f_{\underline{m}}:i \rightarrow m_i.$
\newline
Gives a function of $g: M \times M \times \dots \times M \rightarrow \text{Fun }(N, M); \text{ } g: \underline{m} \rightarrow f_{\underline{m}}.$
\newline
\newline
Note for $f \in  \text{Fun }(N, M)$ define $\underline{b}_{f} := (f(1), f(2), \dots, f(n)) \in M \times M \times \dots \times M (n \text{ times }),$
\newline
This gives a function $h: \text{Fun }(N, M) \rightarrow M \times M \times \dots \times M.$
\newline
Now (check) $h \circ g = id_{M \times M \times \dots \times M}$ and $g \circ h = id_{M \times M \times \dots \times M},$
\newline
\newline
so $h$ is a bijection with inverse $g$.
\newline
\newline
Therefore, $\text{ | } \text{Fun }(N, M) \text{ | } = \text{ | } M \times M \times \dots \times M \text{ | } = m^n,$ (by defition of multiplication principle).
\newpage
\subsection{Ordered selection without repetition}
\textbf{Example. }
\newline
How many four letter words in the letter $A, B, C, D$ can be made with each letter appearing atmost once?
\newline
\newline
$A) $ Choices for first letter: $4$
\newline 
Choices for second letter: $3$
\newline 
Choices for third letter: $2$
\newline 
Choices for fourth letter: $1$
\newline
Total = $ 4 \times 3 \times 2 \times 1.$
\newline
\newline
Alternatively, we can think of the function as $w: \{ 1, 2, 3, 4\} \rightarrow \{ A, B, C, D\},$
\newline
$i \rightarrow \text{ letter in the } i \text{th position },$ the added restriction translates into the condition that is $w$ is bijective.
\newline
\newline
\begin{theorem}
  Rearranging distinct objects: Given finite set $N$ so that: $\text{ | } N \text{ | } = n$, the number
  of bijections $f: N \rightarrow N$ is given by $\text{ | } \text{ Bij}(N, N) \text{ | } = n!.$
\end{theorem}
\textbf{Proof. } Follows from a theorem that is to come (skip for now.)
\newline
\newline
\textbf{Example. } How many injections are there from $\{ 1, 2\} \rightarrow \{ A, B, \dots, Z\}$ ?
\newline
\newline
$A) $ Asking for two letter words with one letter appearing more than once.
\newline
Choices for first letter: $26$
\newline
Choices for first letter: $25$
\newline
Total $= 26 \times 25$
\newline
\newline
\begin{theorem}
  Ordered selection without repetition: Let $R$ and $N$ be finite sets so that $\text{ | } R \text{ | } = r, \text{ | } N \text{ | } = n$ and $r \leq n.$ 
  \\ 
  Number of injections: $f: R \rightarrow N$ is given by $\text{ | } \text{Inj}(R, N) \text{ | } = \frac{n!}{(n + 1)!}$
\end{theorem}
\textbf{Note. } Convention $0! = 1.$
\newline
\newline
\textbf{Notation. } We use $P(n ,r)$ to denote the number 
$$P(n, r) = \frac{n!}{(n - r)!}$$ 
$(n - r)! = $ first $r$ terms.
\newline
\newline
\textbf{Proof of Ordered selection without repetition. }
\newline
By induction on $\text{ | } N \text{ | }, f\text{ | } N \text{ | } = 1, $ then $\text{ | } R \text{ | } = 1$ and $\text{ | } \text{Inj}(R, N) \text{ | } = 1.$
\newline
Assume result is true for $\text{ | } N \text{ | } = K$ and consider the case $\text{ | } N \text{ | } = K + 1.$
\newline
Note the result depends on sizes of $R$ and $N$ so assume $R = \{1, 2, 3\}, r \leq k + 1, N = \{ x_1, x_2, \dots, x_{k + 1}\}.$
\newline
\newline
$$S = \text{Inj}(R, N) = \{f: R \rightarrow N \text{ | } f \text{ is injective} \}.$$
\newline
\newline
Consider subsets: $S_i := \{ f \in S \text{ | } f(i) = x_{k + 1}\}, S_{x_{k + 1}} := \{ f \in S \text{ | } f(i) \not = x_{k + 1} \forall i \in \mathbb{R}\}.$
\newline
\newline
\textbf{Observe. } $S = S_1 \cup S_2 \cup \dots \cup S_{x_{k + 1}}, S_i \cap S_j = \null$ for $i \not = j$ and $S_i \cap S_{x_{k + 1}} = \null.$ (injective)
\newline
Therefore, by addition principle,
\newline
\newline
\boxed{\text{ | } S \text{ | } = \text{ | } S_1 \text{ | } + \text{ | } S_2 \text{ | } + \dots + \text{ | } S_r \text{ | } + \text{ | } S_{x_{k + 1}}\text{ | }}
\newline
\newline
Now, compute $S_i.$ (bijective to $\{ f: R - \{i \} \rightarrow N - \{x_{k + 1}\}\}$), inductions hypothesis $\text{ | } S_i \text{ | } = \frac{K!}{(K - (r - 1))!}$. ($\text{ | } N - x_{k + 1}\text{ | } = K.$)
\newline
\newline
Note bijection of $\text{ | } S_{x_{k + 1}} \text{ | } \text{ bijection with injective } f; \{ f: R \rightarrow N - x_{k + 1} \}$ but $\text{ | } N - x_{k + 1}\text{ | } = K$ so induction hypothesis gives $\text{ | } S_{x_{k + 1}} \text{ | } = \frac{K!}{(K - r)}!$
\newline
\newline
Putting everything together:
$$\text{ | } S \text{ | } = \text{ | } S_1 \text{ | } + \text{ | } S_2 \text{ | } + \dots + \text{ | } S_r \text{ | } + \text{ | } S_{x_{k + 1}}\text{ | },$$ 
$$ = r \frac{K!}{(K - (r - 1))!} + \frac{K!}{(K - r)!},$$ 
$$= \frac{K!(K + 1)}{((K + 1) - r)!} = \frac{(K + 1)!}{((K + 1) - r)!}.$$ 
\newpage
\subsection{Example ordered selection without repetition}
\textbf{Example. } A group fo $13$ pirates are to take a team photo. Photogrpaher asks them to 
organise so that $6$ of them sit in a row of seats while the other $7$ stand behind them. If we do not order
who sits on which seat, how many different arrangements are there?
\newline
\newline
$A) $ Choices for seat $1$: $13$
\newline
Choices for seat $2$: $12$
\newline
\dots
\newline
Choices for seat $6$: $7$
\newline
Total = $P(13, 6) = \frac{13!}{6!}$
\newline
\newline
$Q2) $If the chief and his second in charge must sit in the middle two seats, how many different arrangements are there?
\newline
\newline 
$A) $ Choices for seat $1$: $11$
\newline
Choices for seat $2$: $10$
\newline
Choices for seat $3$: $2$
\newline
Choices for seat $4$: $1$
\newline
Choices for seat $5$: $9$
\newline
Choices for seat $6$: $8$
\newline
Total = $ 2 \times P(11, 4) = 2 \times \frac{11!}{4!}.$
\newpage
\subsection{Remark challenging counting problems}
\textbf{Example. } For a number $n$ a partition is a set of numbers $a, \dots, a_k$ so that $a_1 + a_2 + \dots + a_n = n.$ (ie $4 = 4 = 3 + 1 = 2 + 2 = 2 + 2 + 1 \dots; 5$ partitions.)
\newline
What is the number of partitions in $557$?
\newline
\newline
$A) $ find a pattern and then you can get the number of partitions for $557$.
\newpage
\subsection{Examples counting up to symmetries}
\textbf{Example. } How many words can you make by rearranging the letters in the word "MOON"?
\newline
\newline
$A) $ First, distinguish that there are two "O"s in "MOON". There are $4!$ such rearrangements but they come in pairs. (swap one O with the other O).
\newline
$MO_1NO_2$ then $MO_2NO_1$.
\newline
Therefore, if we were to disregard the difference between $O_1$ and $O_2$, then we get $\frac{4!}{2!}.$
\newline
\newline
\textbf{Example. } How many words can be made by rearranging the letters in "CHEESE"? 
\newline
\newline
$A) $ Find there are 3 "E"s in "$CHE_1E_2SE_3$". 
\newline
We have $6!$ words. But be careful that there can be combinations such as $CHE_1E_2SE_3 \sim CHE_2E_1SE_3$ and etc. 
\newline
\newline
\textbf{Note. }
\newline
$\bullet $ the relation $\sim$ on the artificial words is an equivalence relation. 
\newline
$\bullet $ Elements in any given equivalence class are in one-to-one correspondence with bijectors $f: \{ E_1, E_2, E_3 \} \rightarrow \{ E_1, E_2, E_3\}$, so each equivalence
class contains elements. 
\newline
\newline
Therefore, number of rearrangements of "CHEESE" is $\frac{6!}{3!}.$
\newpage
\subsection{Unordered selection without repetition}
\begin{theorem}
  Unordered selection without repetition: Given a finite set $\text{ | }N\text{ | } = n$ and an
  integer $r$ with $0 \leq r \leq n$, the number of subsets $R \subset N$ so that $\text{ | }R\text{ | } = r$ is
  given by:
  $$\binom{n}{r} = \frac{n!}{(n - r)!r!},$$ where $\binom{n}{r}$ is known as $n$ choose $r$.
\end{theorem}
\textbf{Proof. } Take $R$ to be a set so that $\text{ | }R\text{ | } = r$. 
\newline
Consider $S := \{ f: R \rightarrow S \text{ | } f \text{ is injective}\}$ and define a relation $\sim$ on $S$:
$$f \sim g \text{ iff image}(f) = \text{image}(g).$$
$(i) $ Note $\sim$ is an equivalence relation 
\newline
$(ii) $ $\{\text{equiv classes of }\sim \rightarrow \text{subsets of } N \text{of size } r\}$
$$f: R \rightarrow S \rightarrow \text{image}(f)$$ is bijective.
\newline
$(iii) $ $f \sim g$ iff there is a unique bijection $h: R \rightarrow R$ so that $f \circ h = g$.
$$\text{Subset of} N \text{of size} r = \frac{\not = \text{injections } R \rightarrow S}{\not = \text{bijections } R \rightarrow S} = \frac{n!}{(n - r)!r!}.$$
\newpage
\subsection{Examples unordered selection without repetition}
\textbf{Remark. } Counting subsets of a finite set is counting number of unordered elements in the same subset.
\newline
\newline
\textbf{Example. } How many $13$ card hands have exactly $7$ of one suit and $6$ of another?
\newline
$A) $ = Choices of suits to pick for $7$ cards $\times $ Choices of $7$ unordedred cards in a given suit 
$\times $ choices fo suits to pick six cards from $\times $ choices of $6$ unordered cards in a given suit.
$$ = 4 \times \binom{13}{7} \times 3 \times \binom{13}{6}.$$
\textbf{Example. } How many $13$ card hands contain $4$ cards in each of the two suits and $5$ in another?
\newline
$A) $ Choices of suits to pick $5$ cards from $\times $ number of $5$ cards in a given suit $\times $ choices of suits to pick $4$ cards from 
$\times $ pick four cards from one of the suits $\times $ pick four cards from the other suit
$$ = 4 \times \binom{13}{5} \times \binom{3}{2} \times \binom{13}{4} \times \binom{13}{4} .$$
\newpage
\subsection{Binomial theorem}
\begin{theorem}
  let $n$ be a non negative integer. Then,
  $$(x + y)^n = \sum_{r = 0}^{n} \binom{n}{r} x^ry^{n - r}.$$
\end{theorem}
\textbf{Proof. } (idea)
$$(x + y)^n = (x + y) \times \dots \times (x + y) (n \text{times}).$$
Only way to obtain a term $x^ry^{n - r}$ is to choose $x$ coming from $r$ of these factors. 
\newline
\newline
Corresponds to choosing a subset of size $r$ in a set of size $n$. Therefore, number of such terms in this expression is $\binom{n}{r}.$
\newpage
\subsection{Properties of binomial coefficients}
$(i) $ \textbf{Proposition. } Let $r$ and $n$ be in integers so $0 \leq r \leq n.$ So:
$$\binom{n}{r} = \binom{n}{n - r}.$$
$(ii) $ \textbf{Proposition. } Let $r$ and $n$ be in integers so $1 \leq r \leq n + 1.$ So:
$$\binom{n + 1}{r} = \binom{n}{r} + \binom{n}{r - 1},$$ known as pascal's triangle.
\textbf{Proof.} 
\newline
$$\binom{n}{r} + \binom{n}{r - 1} = \frac{n!}{(n - (r - 1))!(r - 1)!} + \frac{n!}{(n - r)!r!},$$
$$ = \frac{n!r}{(n - (r - 1))!r!} + \frac{n!(n - (r - 1))}{(n - (r - 1)!r!)},$$
$$ = \frac{n!(r + n - r + 1)}{((n - 1) - r)!r!} = \frac{n!(n + 1)}{((n + 1) - r)!r!},$$
$$ = \binom{n + 1}{r}.$$
\newpage
\subsection{Another example of unordered selection without repetition}
\textbf{Example. } How many length $8$ biy strings contain at least $6$ ones?
$i) $ How many such string contain exactly $6$ ones?
$$\binom{8}{6}.$$
$ii) $ How many such string contain exactly $7$ ones?
$$\binom{8}{7}.$$
$iii) $ How many such string contain exactly $8$ ones?
$$\binom{8}{8}.$$
Strings with at least $6$ ones $ = \binom{8}{6} + \binom{8}{7} + \binom{8}{8}.$ 
\newline
\newline
The equivalent is to asking how many b$8$ bit strings contain at most $2$ zeroes.
\newline
$i) $ How many such string contain $2$ zeroes?
$$\binom{8}{2}.$$
$ii) $ How many such string contain $1$ zeroes?
$$\binom{8}{1}.$$
$iii) $ How many such string contain no zeroes?
$$\binom{8}{0}.$$
Strings with at most $2$ zeroes $ = \binom{8}{2} + \binom{8}{1} + \binom{8}{0}.$ 
\newpage
\subsection{Integer solutions to summation equation}
\textbf{Example. } If Tony went to buy 4 scoops of ice cream, in which the flavours available are
pistachio, chocolate and salted caramel, if Tarig would like to try each flavour, how many
possible combinations are there?
\newline
\newline
$A) $ Same as asking number of solutions to $x_1 + x_2 + x_3 = 4, x_i \in \mathbb{N} \cup \{ 0 \}.$
\newline
Assume $x \geq 1.$ Equivalent of asking how many ways we can group $4$ ice creams into $3$ groups. Doing so, we get
$2 + 1 + 1 = 4;$ given by $\binom{4 - 1}{3 - 1} = \binom{3 }{2};$ where $3$ is gaps between sticks and $2$ is dividers required.
\newline
\begin{theorem}
  Let $r \leq n$ be positive integers. Then, the number of positive integers
  ($0$ not included) solutions of
  $$x_1 + x_2 + \dots + x_r = n,$$
  is given by $\binom{n - 1}{r - 1}.$
\end{theorem}
(proof in topic 4 pdf)
\newline
\newline
\begin{theorem}
  Let $r \leq n$ be positive integers. Then, the number of non negative integers
  ($0$ included) solutions to 
  $$y_1 + y_2 + \dots + y_r = n,$$
  is $\binom{n + r - 1}{r - 1}$.
\end{theorem}
\textbf{Proof. } Define $x_i = y_i + 1$ then $x_i \geq 1$. 
\newline
Substituting $y_i = x_i - 1$ into the equations, we get $(y_1 - 1) + (y_2 - 2) + \dots + (y_r - 1) = n,$
\newline
equivalent to
$$y_1 + y_2 + \dots + y_r = n + r.$$
\newpage
\subsection{Example integer solutions}
\textbf{Note. }Counting solutions is not the same as counting portions of $n$.
\newline
\newline
\textbf{Example. } How many different solutions are there to:
$x_1 + x_2 + x_3 = 11$ so that $x_i \in \mathbb{Z}$ and 
\newline
\newline
$(i) $ $x_i \geq 0, n = 11, r = 3$
\newline
$A) $ $\binom{11 + 3 - 1}{3 - 1} = \binom{13}{2}.$
\newline
$(ii) $ $x_i \geq -5$. Substitute $x_r = y_i - 5$ 
\newline
$A) $ $y_1 + y_2 + y_3 = 26, y_i \geq 0.$
\newline
$$\binom{28}{2}.$$
$(iii) $ $0 \leq x_i \leq 5$ substitute $x_i = -y_i + 5$
\newline
$y_1 + y_2 + y_3 = 4, 0 \leq y_i \leq 5.$
$$ \binom{6}{2}. (\text{ignore }5)$$
$(iv) $ $0 \leq x_i \leq 7$ substitute $x_i = -y_i + 7$
\newline
$y_1 + y_2 + y_3 = 10, 0 \leq y_i \leq 7.$
Here, theorem doesn't apply.
\newpage
\subsection{Inlcusion - Exclusion Principle}
\begin{theorem}
  Inclusion - Exclusion Principle:  Let $A_1, \dots, A_n$ be a finite 
  list of finite sets. Define:
  $$S_1 = \sum_{i} = \text{ | } A_i \text{ | },$$
  $$S_2 = \sum_{i_1 < i_2} = \text{ | } A_{i_1} \cap A_{i_2} \text{ | },$$
  $$ \dots $$
  $$S_n = \text{ | } A_{i_1} \cap \dots \cap A_{i_n} \text{ | }.$$
  \newline
  \newline
  Then, $\text{ | } A_1 \cap \dots \cap A_n \text{ | } = \sum_{r = 1}^{n} (-1)^{r + 1} S_k.$
\end{theorem}
\textbf{Remark. } Refinement of the addition principle. 
\newline
\newline
[proof in topic 4 pdf]
\newpage
\subsection{Example inclusion exclusion principle}
\textbf{Example. } A card game requires us to remove
the (K clover) card from the standard $52$ pack. Then, 
each of the $3$ players is dealt $17$ cards.
\newline
\newline
$a) $ How many $17$ card hands contain exactly 
$8$ clubs?
\newline
$A) $ Choices of $8$ clubs from $12$ available $\times $ Choices of $9$ cards from the rest
$$ \binom{12}{8} \times \binom{39}{9}.$$
$b) $ How many $17$ card hands contain exactly $8$ hearts?
\newline
$A) $ Choices of $8$ hearts from $13$ available $\times $ Choices of $9$ cards from the rest
$$\binom{13}{8} \times \binom{38}{9}.$$
c) How many $17$ card hands contain exactly $8$ cards of at least one suit?
\newline
$A) $ $A_1 = \{17 \text{ card hands with exactly $8$ clubs} \},$
$$A_2 = \{17 \text{ card hands with exactly $8$ diamonds} \},$$
$$A_3 = \{17 \text{ card hands with exactly $8$ hearts} \},$$
$$A_4 = \{17 \text{ card hands with exactly $8$ spades} \}.$$
\textbf{note. } $\text{ | } A_2 \text{ | } = \text{ | } A_3 \text{ | } = \text{ | } A_4 \text{ | }.$
\newline
\newline
Consider:
$$\text{ | } A_1 \cap A_2 \text{ | } = \text{ | } A_1 \cap A_3 \text{ | } = \text{ | } A_1 \cap A_4 \text{ | },$$
$$ = \binom{12}{8} \binom{13}{8} \binom{26}{1}.$$
$$\text{ | } A_2 \cap A_3 \text{ | } = \text{ | } A_2 \cap A_4 \text{ | } = \text{ | } A_3 \cap A_4 \text{ | },$$
$$ = \binom{13}{8} \binom{13}{8} \binom{25}{1}.$$
\textbf{Note. } $\text{ | } A_i \cap A_j \cap A_k \text{ | } = 0.$
\newline
Inclusion-Exclusion gives the answer: 
$$\text{ | } A_1 \text{ | } + \text{ | } A_2 \text{ | } + \text{ | } A_3 \text{ | } + \text{ | } A_ 4 \text{ | } - \text{ | } A_1 \cap A_2 \text{ | } \dots - \text{ | } A_3 \cap A_4 \text{ | },$$
$$\binom{12}{8}\binom{39}{9} + 3\binom{13}{8}\binom{38}{9} - 3(\binom{12}{8} \binom{13}{8} \binom{26}{1}) - 3(\binom{13}{8} \binom{13}{8} \binom{25}{1}).$$
\newpage
\subsection{Example inclusion exclusion and integer solutions}
\textbf{Example. } How many integer solutions to $x_1 + x_2 + x_3 = 11, (*)$ for $0 \leq x_i \leq 7$ ?
\newline
\newline
$A) $ Let $A := \{ (x_1, x_2, x_3) \in \mathbb{Z} s.t. (*) \text{ and } x_i \geq 0\},$
$$B:= \{ (x_1, x_2, x_3) \in \mathbb{Z} s.t. (*) \text{ and } 0 \leq x_i \leq 7\},$$
$$A_1 := \{ (x_1, x_2, x_3) \in \mathbb{Z} s.t. (*) \text{ and } x_1 \geq 8, x_2 \geq 0, x_3 \geq 0\},$$
$$A_2 := \{ (x_1, x_2, x_3) \in \mathbb{Z} s.t. (*) \text{ and } x_1 \geq 0, x_2 \geq 8, x_3 \geq 0\},$$
$$A_3 := \{ (x_1, x_2, x_3) \in \mathbb{Z} s.t. (*) \text{ and } x_1 \geq 0, x_2 \geq 0, x_3 \geq 8\}.$$
Then $B = A `(A_1 \cup A_2 \cup A_3).$
\newline
Now $\text{ | } A \text{ | } = \binom{13}{2}.$
\newline
\newline
We may compute $\text{ | } A_1 \cup A_2 \text{ | }$ using the inclusion-exclusion principle.
\newline
Note $\text{ | } A_i \cap A_j \text{ | } = \null,$
\newline
Since $16 > 11,$
\newline
\newline
Therefore, $\text{ | } A_1 \cup A_2 \cup A_3 \text{ | } = \text{ | } A_1 \text{ | } + \text{ | } A_2 \text{ | }  + \text{ | } A_3 \text{ | },$
Also $\text{ | } A_i \text{ | } = \binom{5}{2}; \text{ | } B \text{ | } =\binom{13}{2} - 3\binom{5}{2}.$
\newpage
\subsection{Aside topological inclusion exclusion}
\textbf{Aside: Euler characteristic} (topological inclusion-exclusion)
\newline
\newline
Given a convex polyhedron (eg cube, tetrahedron, \dots ); compute:
$$X = V - E + F,$$
where $V = \text{vertices}, E = \text{edges and} F = \text{faces}$ (always a polyhedron).
\newline
\newline
\textbf{Example. } 
\newline
$1) $ Tetrahedron: $x = 4 - 6 + 4 = 2,$
\newline
$2) $ Cube: $x = 8 - 12 + 6 = 2.$
\newline
$3) $ Square - based pyramid $x = 5 - 8 + 5 = 2.$
\newpage
\subsection{Miscellaneous Examples}
\textbf{Examples. }
\newline
$1) $ A spider has one sock and one shoe for each of its $8$ feet. On each foot the sock 
must go on before the shoe. How many ways can the spider put his shoes and socks on?
\newline
\newline
$A) $ $16 $ digit string with digits $1 \dots 8$ with each digit appearing exactly twice. 
\newline
\newline
e.g. $1_1571_2678223_3\dots,$ where $1, 2, 3$ as subscripts are sock number and shoe on foot.
\newline
\newline
On the other other hand, each string defines a unique sock-shoe sequeunce.
\newline
\newline
Sock-shoe sequeunces $= 16$ digit strings as above $= \frac{16!}{(2!)^8}.$
\newline
\newline
$2) $ How many solutions are there to $x_1 + x_2 + x_3 \leq 13$ with $x_i \geq 0,$
\newline
\newline
$A) $ Define $x_4 := 13 - (x_1 + x_2 + x_3)$ this by defnitions $\geq 0.$
\newline
Every solution to $x_1 + x_2 + x_3 \geq 13; x_i \geq 0,$
\newline
Corresponds to a unique solution To
$$x_1 + x_2 + x_3 + x_4 = 13; x_i \geq 0,$$
and vice versa. Hence, the answer is $\binom{13 + 4 - 1}{3} = \binom{16}{3}.$
\newline
\newline
$3) $ $6$ students and $3$ prof. are to sit in a row of $9$ seats. The Prof.s 
come into the room first. How many ways (order of profs matter) can the profs 
sit so that there is a student either side of each of them. 
\newline
\newline
$A) $ Imagine the students lined up in a row. One needs to fit the $3$ profs in the gaps between them.
There are $5$ gaps and the order matters. $\binom{5}{3} \times 3! (\text{ordering})$.
\newline
\newline
$4) $ What is the coefficient of $x^11y^49z^40$ in $(x + y + z)^100 ?$
\newline
\newline
$A) $ This is equivalent to counting number of $100$ letter words in $x, y, z$ with exactly $11 x$s, $49 y$s and $40 z$s which is: 
$$\frac{100!}{11!49!40!}.$$
$5) $ What is the coefficient of $x_1^{\alpha_1}x_2^{\alpha_2} \dots x_k^{\alpha_k}$ in $(x_1 + x_2 + \dots + x_k)^n$ where
$n = \alpha_1 + \alpha_2 + \dots + \alpha_k?$
\newline
$A) $ $\frac{n!}{\alpha_1!\alpha_2! \dots \alpha_k!}$
\newline
\newline
$6) $ What is the coefficient of $a^{33}b^{289}c^4d^{71}$ in $(2a + 4b - 3c + 7d + e)^{1111}?$
\newline
\newline
$A) $ $\frac{2^{32}4^{289}(-3)^{4}7^{71} 1111!}{37!289!4!71!}.$ 
\newline
$7) $ What is the coefficient of $x^{11}$ in $(1 + x + x^2 + \dots + x^{11})^3?$
\newline
\newline
$A) $ This is equivalent to solving $y_1 + y_2 + y_3 = 11, 0 \leq y_i \leq 11$; 
$$\binom{11 + 3 - 1}{3 - 1} = \binom{13}{2}.$$
\newpage
\subsection{The pigeonhole principle}
\begin{theorem}
  The Pigeonhole Principle: gently place $n + 1$ pigeons in $n$ pigeonholes 
  then there is at least one pigeonhole with two or more pigeons. 
  \newline
  \newline
  Let $A$ be a set of size $r$ and $A_1 \dots A_n \subset A$ be mutually
  disjoint s.t $A = A_1 \cup \dots \cup A_n.$ If $r < n$ then 
  $\exists i s.t. \text{ | } A_i \text{ | } \geq 2.$
\end{theorem}
(proof in topic 4: corollary)
\newpage
\subsection{Examples PHP (Pigeon Hole Principle)}
\textbf{Examples. }
\newline
\newline
$1) $ A martian has an infinite supply of red, green, blue and yellow
socks in his drawer. How many does he need to draw out in order to guarentee
he has a pair?
\newline
\newline
$A) $ Pigeonhole principle (PHP) ensures he has $2$ socks of the smae colour if draws $ > $ colours $ = 4$.
Therefore, the smallest such possible number of draws is $5.$
\newline
\newline
$2) $ Take $S \subset \mathbb{Z}$ s.t. $\text{ | } S \text{ | } = n + 1.$ Show that 
$\exists s_1, s_2 \in S$ s.t. $n \text{ | } (S_2 - S_1).$
\newline
\newline
$A) $ Consider the function $f: S \rightarrow \{ 0, 1, 2, \dots , n - 1\}$ given by taking residue 
$\text{ mod } n$. Then since $\text{ | } S \text{ | } > \text{ | } \{ 0, 1, \dots , n - 1\}\text{ | }$. The PHP gives 
gives that $f$ is not injective. So, $\exists s_1 \not = s_2$ s.t. $f(s_1) = f(s_2).$
$$ => (s_2 - s_1) = 0 (\text{mod} n),$$
$$ => n \text{ | } (s_2 - s_1).$$
$3) $ Pick $4$ distinct integers from $\{ 1, 2, \dots, 6 \}$ then there must be $2$ of them whose sum is $7$.
\newline
\newline
$A) $ \textbf{Proof. } Split $\{ 1, 2, \dots , 6\}$ into subsets 
$$\{ 1, 6\}, \{ 2, 5\}, \{ 3, 4 \} (\star).$$
Note that the elements in each of these sum up to $7$. Since $4 > 3$, the PHP gives that at least two
of the numbers picked must live in one of the subsets in $(\star)$. Hence, the result.
\newline
\newline
$4) $ In a party of $20$ people, at least two have the same number of friends among the $20$. 
\newline
\newline
$A) $ In such a party there can't be two people who know $0$ and $19$ people, respectively. 
\newline 
Therefore, the number of possibilities for friends is $19$. 
\newline
Since $20 > 19,$ the PHP then gives us that there are at least two people with the same number of friends.
\newline
\newline
$5) $ Take the square of side length $2$. Place $5$ points on it at random. Show that they must lie within $sqrt(2)$ of each other.
\newline
\newline
$A) $ Split the squares as above into $6$ regions. Then, two points must lie in the same region 
by PHP. Noting that the distance between any two points in the same region is $\leq sqrt(2)$ gives the result. 
\newline
\newline
$6) $ Choose $n + 1$ numbers from $\{ 1, 2, \dots 2n \}$ show that two of the chosen must be coprime. 
\newline
\newline
$A) $ Split the set $\{1, 2, \dots 2n \}$ into $n$ subsets $\{1, 2\}, \{3, 4\} \dots, \{2n - 1, 2n\}$  then at least
two will live in the same subset $\{k, k + 1\}$ but $\text{gcd}(k, k + 1) = 1$. This gives the result.
\newpage
\subsection{Generalised pigeon hole} 
\begin{theorem}
  Let $f: A \rightarrow B$ be a function so that $\text{ | } A \text{ | } = r$ and $\text{ | } B \text{ | } = n.$ 
  For $b \in B$ let 
  $$f^{-1}(b) := \{ a \in A \text{ | } f(a) = b\}.$$
  Then $\exists b \in B \text{ s.t. },$
  $$\text{ | } f^{-1}(b) \text{ | } \geq \lceil \frac{r}{n} \rceil.$$
\end{theorem}
\textbf{Example. } Amongst $100$ people at least $9$ are born in the same month.
\newline
\newline
$A) $ $A := \{ \text{people} \}: \text{ | } A \text{ | } = 100; B := \{\text{months of the year}\}: \text{ | } B \text{ | } = 9.$
$$f: A \rightarrow B,$$
$$\text{person} \rightarrow \text{birth month},$$
$$\text{Since } \lceil \frac{100}{12} \rceil = 9,$$
GPHP gives $\exists$ month so that there are at least $9$ people born in it. 
\newpage
\subsection{Examples GPHP (Generalised Pigeon Hole Principle)}
\textbf{Examples. } 
\newline
\newline
$1) $ $33$ squares in a standard $8 \times 8$ chess board are coloured red. Show that there must be $3$ that form
an L-shape. 
\newline
\newline
$A) $ Divide the board into $16$ regions of size $2 \times 2$.
\newline
The GPHP gives that there is a region with at least $\lceil \frac{33}{16} \rceil = 3,$ colored squares.
\newline
Therefore, the region must contain an L-shape.
\newline
\newline
$2) $ Prof. Craw wants all students in his class to have the same birthday. He is not allowed to have less than $30$ students. What is the least number 
of students that must enroll for him to guarentee that $30$ have the same birthday.
\newline
\newline
$A) $ By GPHP we seek x s.t. 
$$\lceil \frac{x}{366} \rceil = 30,$$
$$x = (366 \times 29) + 1\text{students.}$$
\newpage
\subsection{Warm up and notation recurrence}
\textbf{Recurrence. } Motivating Example:
\newline
\newline
What is the number of strings of length $n$ in $0$s and $1$s that don't contain `$111$` as a subword?
\newline
$A) $ Any word as above must end in either $0, 01, 011$. Deleting the possible endings gives us words of length $n - 1, n - 2, n - 3$ that do not contain $111$. If $a_n$ the answer we seek, 
then $a_n = a_{n - 1} + a_{n - 2} + a_{n - 3}.$
\newline
\newline
\begin{theorem}
  Let $a_n$ be a sequence of natural numbers. An equation expressing $a_n$ in terms of $a_{n - 1}, a_{n - 2}, a_{n - 3}, \dots, a_{n - k}$
  is called reccurence relation of order $k$.
\end{theorem}
To determine sequence, use the first few terms. 
\newline
\newline
\textbf{Example. } The fibonacci numbers $F_n$ are defined by the following recurrence
relation: $F_n = F_{n - 1} + F_{n - 2}$ with initial conditions $F_0 = 0, F_1 = 1.$
\newline
\newline
$A) $ A general recurrence relation of order $k$ looks like 
$$a_n + c_1(n)a_{n - 1} + \dots + c_k(n)a_{n - k} = f(n).$$
here, $c_i(n)$ and $f(n)$ are arbitary functions. We assume $c_k(n) \not = 0,$
\newline
$\bullet $ the recurrence is linear if $c_i(n)$ are constant functions. 
\newline
$\bullet $ the recurrence is homogenous if $f(n) = 0.$
\newpage
\subsection{First and second order linear}
\begin{theorem}
  The linear homogenous recurrence of order 1
  $$a_n - ca_{n - 1} = 0 \text{with} a_0 = A,$$
  has soution $a_n = Ac^n.$
\end{theorem}
\begin{theorem}
  Consider the second order linear homogenous recurrence $a_n + pa_{n - 1} + qa_{n - 2} = 0,$ and let $\alpha, \beta$ be solutions to
  $$r^2 + pr + q = 0.$$
  Then,
\end{theorem}
$\bullet $ if $\alpha, \beta$ are real and $\alpha \not = \beta$ the general solution
is 
$$a_n = A\alpha^n + B\beta^n.$$
\newline
$\bullet $ if $\alpha = \beta$ then the general solution Is
$$a_n = A\alpha^n + Bn\beta^n.$$
\newline
$\bullet $ if $\alpha, \beta$ are complex conjugates (not examinable) i.e. $\alpha_1\beta = pe^{+- i\theta}$ then
the general solution is 
$$a_n = p^n(A\text{ cos}n\theta + B\text{ sin}n\theta).$$
Here, $A$ and $B$ are constants to be determined by initial conditions. 
\newline
\newline
The polynomial $r^2 + pr + q = 0$ associated to a second order linear occurence called characteristic polynomial.
\newline
\newline
\textbf{Remark. } The same technique can be applied to linear homogenous recurrences
of arbitary order. Use the coefficients to write down a characteristic polynomial. Compute roots.
\newline
\newline
General solution?
\newline
\newline
$a) $ Consider $a_n + pa_{n - 1} + qa_{n - 2} = 0.$
\newline
If we assume the solution is the form of $a_n = A \alpha^n,$ then substituting in we get:
$$A \alpha^n + p(A \alpha^{n - 1}) + q(A \alpha^{n - 2}) = 0,$$
$$A \alpha^{n - 2} (\alpha^2 + p\alpha + q) = 0.$$
but if $A \alpha^{n - 2} \not = 0,$
$$\alpha^2 + p\alpha + q = 0.$$
in other words, $\alpha$ is a root of the characteristic polynomial.
\newpage
\subsection{Examples homogeneous}
\textbf{Examples. }
\newline
\newline
$1) $ Solve $a_n - 2a_{n - 1} - 3 = 0,$ with $a_0 = 0, a_1 = 4.$
\newline
\newline
$A) $ characteristic equation: $r^2 - 2r - 3 = 0.$
\newline
\newline
It has roots $r = 3, r = -1.$ Take $\alpha = 3, \beta = -1$ then general solution is $a_n = A(3^n) + B(-1^n).$
\newline
Now, $a_0 = (A)3^0 + B(-1)^0 = A + B = 0.$
\newline
$a_1 = (A)3^1 + B(-1)^1 = 3A - B = 4.$
\newline
\newline
Solving simultaneously, we bet $A = 1, B = -1.$
\newline
\newline
$2) $ Solve fibonacci reccurence $F_{n + 2} = F_{n + 1} + F_n, F_0 = 0, F_1 = 1.$
\newline
\newline
$A) $ Rewrite this as $F_{n + 2} - F_{n + 1} - F_n = 0.$
\newline
The characteristic equation is $r^2 - r - 1 = 0.$
\newline
\newline
It has solutions $\alpha_1 \beta = \frac{1 +- \sqrt{5}}{2},$ the general solution being: $a_n = A(\frac{1 + \sqrt{5}}{2})^n + B(\frac{1 - \sqrt{5}}{2})^n.$
\newline
\newline
$$F_0 = 0 => A + B = 0,$$
$$F_1 = 1 => A(\frac{1 + \sqrt{5}}{2}) + B(\frac{1 - \sqrt{5}}{2}),$$
where solving simultaneously gives $A = \frac{1}{sqrt{5}}, B = -\frac{1}{sqrt{5}}.$
\newline
\newline
Therefore, $F_n = \frac{1}{\sqrt{5}}(\frac{1 + \sqrt{5}}{2})^n - \frac{1}{\sqrt{5}}(\frac{1 - \sqrt{5}}{2})^n.$
\newline
\newline
$3) $ Solve $a_n = 6a_{n - 1} - 9a_{n - 2}$ with $a_0 = 4, a_1 = 11.$
\newline
\newline
$A) $ Characteristic polynomial is 
$$r^2 - 6r + 9 = 0,$$
$$ie \text{ } (r - 3)^2 = 0.$$
So, $\alpha = \beta = 3,$ the general solution is $a_n = A(3^n) + B(n)(3^n).$
$$a_0 = 4 => A(3^0) + B(0)(3^0) = A = 4,$$
$$a_1 = 11 => A(3^1) + B(1)(3^1) = 3(A) + 3(B) = 3(4) + 3B = 1.$$
Making $B = -\frac{1}{3}$. Therefore, $a_n = 4(3^n) + (-\frac{1}{3})(n)(3^n).$
\newline
\newline
$4) $ Solve 
$$a_n = -5a_{n - 1} - 6a_{n - 2}.$$
with $a_0 = 2, a_1 = 1.$
\newline
\newline
$A) $ Characteristic polynomial is $a_n = A(-3^n) + B(-2^n).$
$$r^2 + 5r + 6 = 0,$$
$$(r + 3)(r + 2) = 0,$$
general solution $a_n = $
$$a_0 = A(-3^0) + B(-2^0) = A + B = 2,$$
$$a_1 = A(-3^1) + B(-2^1) = -3A - 2B = 1,$$
Solving simultaneously gives $A = -5, B = 7.$ Therefore, $a_n = -5(-3^n) + 7(-2^n).$
\newpage
\subsection{Non-homogeneous recurrence}
\begin{theorem}
  The general solution of a second order, non-homogenous solution, linear occurence
  $$a_n + 9a_{n - 1} + ra_{n - 2} = f(n) -> (1).$$
  is $a^n$ the form
  $$a_n = h_n + p_n.$$
  where $h_n$ is the general solution to the homogenous reccurence relation
  $$a_n + 9a_{n - 1} + ra_{n - 2} = 0 -> (2).$$
  and $p_n$ is some 'particular' solution to the equation (1).
\end{theorem}
(proof in topic $4$ lecture $4.29$).
\newpage
\subsection{Examples non-homogeneous}
\textbf{Example. } Solve $a_n - a_{n - 1} - 6a_{n - 2} = 12$ with $a_0 = 1, a_1 = 2.$
\newline
\newline
$A) $ Theorem says $a_n = h_n + p_n$ where $h_n$ is solution to $a_n - a_{n - 1} - 6a_{n - 2} = 0.$
\newline
\newline
Whose general solution is
$$h_n = A(3^n) + B(-2^n),$$
For $p_n$ gues $p_n = d.$
\newline
Substituting we get 
$$d - d - 6d = 12.$$
Therefore, $p_n = d = -2.$
\newline
Out recurrence has a general solution
$$a_n = h_n + p_n,$$
$$ = A(3^n) + B(-2^n) - 2.$$
Now, 
$$a_0 = 1 => A(1) + B(1) - 2 = 1,$$
$$a_1 = 2 => A(3) + B(2) - 2 = 2.$$
Which gives $A = 2, B = 1.$
\newline
\newline
Therefore, $a_n = 2(3^n) + (-2^n) - 2.$
\newline
\newline
\textbf{Remark } To find the values of the constants $A$ and $B$ in the 
general solution, we substitute the initial condition into $a_n = h_n + p_n$ and 
not $h_n$ only. 
\newline
\newline
\textbf{Example. } Find the general solution to $a_n - a_{n - 1} - 6a_{n - 2} = 36n.$
\newline
\newline
$A) $ First compute $h_n$ ie a general solution to $a_n - a_{n - 1} - 6a_{n - 2} = 0,$
$$h_n = A(3^n) + B(-2^n).$$
Since $f(n) = 36n$ is a polynomial of degree $1$ we guess $p_n = cn + d.$
\newline
Substituting we get
$$(cn + d) - (c(n - 1) + d) - 6(c(n - 2) + d) = 36n,$$
$$(c - c - 6c)n + (d + c - d + 12c - 6d) = 36n,$$
$$ie \text{ } (-6c)n + (13c - 6d) = 36n + 0.$$
Therefore, $c = -6.$
$$13() - 6d = 0 => d = -13.$$
Therefore, the general solution is $a_n = A(3^n) + B(-2^n) + (-6)n - 13.$
\newline
\newline
\textbf{Example. } Solve $a_n + 5a_{n - 1} - 6a_{n - 2} = (-6)^n$ with $a_0 = 12, a_1 = \frac{-1}{7}.$
\newline
\newline
$A) $ 
$$h_n = A(-6^n) + B(1^n).$$
For $p_n,$ guess $p_n = c(-6^n).$ (won't work since $p_n$ is a solution to the homogenous part.)
\newline
\newline  
Revised guess $p_n = cn(-6^n),$
\newline
Substituting, we get:
$$cn(-6^n) + 5c(n - 1)(-6)^{n - 1} - 6c(n - 2)(-6^{n - 2}) = -6^n.$$
Simplifying and solving we get,
$$p_n = \frac{6}{7} (-6^n).$$
Therefore, general solution $a_n = A(-6^n) + B(1^n) + \frac{6}{7}n(-6^n).$
\newline
Now,
$$a_0 = 12 => A(-6^0) + B(1^0) + \frac{6}{7}(0)(-6^0) = 12,$$
$$a_1 = \frac{-1}{7} => A(-6^1) + B(1^1) + \frac{6}{7}(1)(-6^1) = \frac{1}{7}.$$
$$=> A = 1, B = 11.$$
Therefore, $a_n = (-6^n) + 11(1^n) + \frac{6}{7}n(-6^n).$
\newline
\newline
\textbf{Example. } Find the general solution 
$$a_n - a_{n - 1} - 6(a_{n - 2}) = 2n(3^n). (\star).$$
$A) $ The homogenous part is given by:
$$h_n = A(3^n) + B(-2^n),$$
For particular solution try $p_n = (cn + d) 3^n.$
\newline
\newline
Doesn't work because $3^n$ is already a solution to the homogenous part so 
substituting into $(\star)$ will give $0 = 2n(3^n),$ which is false.
\newline
\newline
Try $p_n = n(cn + d)(3^n)$ instead. Substituting and simplifying, we get 
$$(32c)n3^{n - 2} - (-21c + 5d)3^{n - 2} = 18n3^{n - 2}.$$
Ie
$$32c = 18, -21c + 5d = 0.$$
to get $c = \frac{9}{15}, d = \frac{243}{255}.$
\newline
\newline
Hence, $p_n = n(\frac{9}{15}n + \frac{243}{255})3^n.$
\newline
Therefore, the general solution is $a_n = A(3^n) + B(-2^n) + n(\frac{9}{15}n + \frac{243}{255})(3^n).$
\newpage
\end{document}
