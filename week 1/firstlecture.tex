\documentclass{article}
\usepackage{amsmath,amsthm,amsfonts}                        % AMS Math
\usepackage{thmtools}                                       % Theorem Tools
\usepackage{bm}        
\usepackage{tabto}                                     % Bold Math

\title{MATH1081 notes}
\author{Nira (z5417727)}
\date{September 17th, 2022}
\begin{document}
\maketitle 
\tableofcontents
\newpage
\section{1.1: Introduction}
1. addition, multiplication, division and subtraction
\newline
2. Mainly dealing with finite sets

\newpage
\section{1.2: Sets and subsets}
\boxed{ $\text{A set is a well defined collection of distinct objects}$ }
\newline
\newline
Example: $S = \{1, a, 3\}, A =\{\Pi, 1\}.$
\newline
1. $e \notin A$; it is not in A
\newline
2. For example, if A is a set of all integers; $\{ \text{all even integers} \}$
    = $\{n \in \mathbb{R} | \text{n is even}\}$.
\newline
3. We can remove superfluos items (elements that occur more than one).
$A = \{ 1, 2, 3, 3 \}$ where $3$ can be removed.
\newline
\newline
\newline
Example:
\newline
$A = \{ 1, 2, 3 \},
B = \{2, 3, 1 \}, 
C = \{1, 2, 3, 3 \},
D = \{ 1 , 3 \}.$
\newline
Here, D is a proper subset of A, B, C; A, B, C are supersets of D.
\newline
\boxed {$$\subseteq: \text{Subset (proper subset)}, $$
$$ \supseteq: \text{Superset}.$$}
\newline
\newline
1. To prove if a set is a proper subset; do the following:
\newline
For example, if $D \in A$, then check if $e \in D$
\newline
If $e \in D,$ then $e \in A$.
Thus, it would be a proper subset (here, e is just an element).
\newline
\newline
2. To prove that two sets are equal;
\newline
For example, if A = B, prove:
\newline
i) $A \subseteq B$; if an element is in A, then the element is in B.
\newline
ii) $B \subseteq A$; if an element is in B, then the element is in A.
\newpage
\section{1.3: Power Sets and Stability}
Subsets of $A = \{1, 2, 3\} $:
\newline
1. Could throw everything out to get empty set $\Phi$,
\newline
2. One element each: $\{ 1 \}, \{ 2 \}, \{ 3 \}$,
\newline
3. Two elements: $\{ 1, 2 \}, \{ 2, 3 \}, \{ 1, 3 \}$,
\newline
4. Set itself: $A$.
\newline
The set containing 1, 2, 3, 4 is called the powerset of A.
\newline
\newline
Given $
A = \{1, 2, 3\}, B = \{1, 2, 3, 3\}, C = \{1, 3\}, D = \{1, 3\}$, where \newline
$A = B$, $C \subseteq A, B$ and $D \not \subseteq A, B, C$.
\newline
1. size of A = 3, B = 3, C = 2, D = 2.
\newline
[Exercise with A  = {0, 1, {0, 1}}, B done in word].
\newpage
\subsection{1.4: Set Operations}
Boolean Operators ("not" operation in programming): 
\newline
1. Complement:
\newline
Let there be a set A in U (A: all of the people in the video, U: universal set of everyone in the world, $A^c =$ complement of A).
\newline
\newline
\boxed { $$A^c = \{ x \in U | x \not \in A \}.$$}
\newline
\newline
2. Intersecting ("and" operation in programming):
\newline
If there is $A, B$, intersecting, 
\newline
\newline
\boxed{ $$A \cap B = \{ x \in A | x \in B\}.$$}
\newline
\newline
3. Union ("or" operation in programming):
If there is $A, B$, A or B is:
\newline
\newline
\boxed{ $$A \cup B = \{ x \in U | x \in A \text { or } x \in B\}.$$}
\newline
\newline
4. Difference:
If there is $A, B$, intersecting, 
\newline
\newline
\boxed{ $$A - B = \{ x \in A | x \in B \}.$$}
\newline
\newline
\end{document}