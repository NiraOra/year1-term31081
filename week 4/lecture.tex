\documentclass{article}
\usepackage{amsmath,amsthm,amsfonts}                        % AMS Math
\usepackage{thmtools}                                       % Theorem Tools
\usepackage{bm}        
\usepackage{tabto}                                          % Bold Math

\title{MATH1081 notes}
\author{Nira (z5417727)}
\date{September 17th, 2022}
\begin{document}
\maketitle 
\tableofcontents
\newpage
\section{Topic 1}
\subsection{Introduction}
1. addition, multiplication, division and subtraction
\newline
2. Mainly dealing with finite sets

\newpage
\subsection{Sets and subsets}
\boxed{ $\text{A set is a well defined collection of distinct objects}$ }
\newline
\newline
Example: $S = \{1, a, 3\}, A =\{\Pi, 1\}.$
\newline
1. $e \notin A$; it is not in A
\newline
2. For example, if A is a set of all integers; $\{ \text{all even integers} \}$
    = $\{n \in \mathbb{R} | \text{n is even}\}$.
\newline
3. We can remove superfluos items (elements that occur more than one).
$A = \{ 1, 2, 3, 3 \}$ where $3$ can be removed.
\newline
\newline
\newline
Example:
\newline
$A = \{ 1, 2, 3 \},
B = \{2, 3, 1 \}, 
C = \{1, 2, 3, 3 \},
D = \{ 1 , 3 \}.$
\newline
Here, D is a proper subset of A, B, C; A, B, C are supersets of D.
\newline
\boxed {$$\subseteq: \text{Subset (proper subset)}, $$
$$ \supseteq: \text{Superset}.$$}
\newline
\newline
1. To prove if a set is a proper subset; do the following:
\newline
For example, if $D \in A$, then check if $e \in D$
\newline
If $e \in D,$ then $e \in A$.
Thus, it would be a proper subset (here, e is just an element).
\newline
\newline
2. To prove that two sets are equal;
\newline
For example, if A = B, prove:
\newline
i) $A \subseteq B$; if an element is in A, then the element is in B.
\newline
ii) $B \subseteq A$; if an element is in B, then the element is in A.
\newpage
\subsection{Power Sets and Stability}
Subsets of $A = \{1, 2, 3\} $:
\newline
1. Could throw everything out to get empty set $\Phi$,
\newline
2. One element each: $\{ 1 \}, \{ 2 \}, \{ 3 \}$,
\newline
3. Two elements: $\{ 1, 2 \}, \{ 2, 3 \}, \{ 1, 3 \}$,
\newline
4. Set itself: $A$.
\newline
The set containing 1, 2, 3, 4 is called the powerset of A.
\newline
\newline
Given $
A = \{1, 2, 3\}, B = \{1, 2, 3, 3\}, C = \{1, 3\}, D = \{1, 3\}$, where \newline
$A = B$, $C \subseteq A, B$ and $D \not \subseteq A, B, C$.
\newline
1. size of A = 3, B = 3, C = 2, D = 2.
\newline
[Exercise with A  = {0, 1, {0, 1}}, B done in word].
\newpage
\subsection{Set Operations}
Boolean Operators ("not" operation in programming): 
\newline
1. Complement:
\newline
Let there be a set A in U (A: all of the people in the video, U: universal set of everyone in the world, $A^c =$ complement of A).
\newline
\newline
\boxed { $$A^c = \{ x \in U | x \not \in A \}.$$}
\newline
\newline
2. Intersecting ("and" operation in programming):
\newline
If there is $A, B$, intersecting, 
\newline
\newline
\boxed{ $$A \cap B = \{ x \in A | x \in B\}.$$}
\newline
\newline
3. Union ("or" operation in programming):
If there is $A, B$, A or B is:
\newline
\newline
\boxed{ $$A \cup B = \{ x \in U | x \in A \text { or } x \in B\}.$$}
\newline
\newline
4. Difference:
If there is $A, B$, intersecting, 
\newline
\newline
\boxed{ $$A - B = \{ x \in A | x \in B \}.$$}
\newline
\newline
[examples in word doc]
\newpage
\subsection{The Inclusion-Exclusion Principle}
[example in Word]
\newline
\boxed{ $$|A \cup B| = |A| + |B| - | A \cap B|$$.}
\newline
\newline
For three elements, 
\newline
\newline
\boxed{ $$|A \cup B \cup C| = |A| + |B| + |C| - | A \cap B| - |A \cap C| - |B \cap C| + |A \cap B \cap C|$$.}
\newline
\newline
[example in word]
\newpage
\subsection{Sets Proofs}
[proof question in word]
\newline
Hints for proofs:
\newline
1. To prove that $S \subseteq T$, we can assume that $x \in S$ and show that $x \in T$.
\newline
2. 
To prove that $S = T$, we can show that $S \subseteq T$ and $T \subseteq S$.
\newline
\newline
Scaffold: 
\newline
\boxed{ \text{Proof: Suppose that \dots 
\newline
\dots (proof)
\newline
we see that/ it follows \dots (conclusion) (end with shaded box to indicate end of proof.)}}
\newline
\newline
Note that the "Suppose that" part of the proof is usually whatever the if statment mentions.
\newline
\newline
For example, if the question is "Prove that if $A \cap B = A$, then $A \cup B = B$, then the proof starts like this:
\newline
\boxed{\underline{Proof}: \text{ Suppose that } A \cap B = A. }
\newline
For questions like "is this statement true", there are two ways to approach the question:
\newline
1. If the statement is true (if you think it is true), then prove it.
\newline
2. If the statement is false, then give a counter-example that proves it false. 
\newline
[examples in word]
\newpage
\subsection{Laws of Set Algebra}
\underline{ Laws of Set Algebra}
\newline
$$1\text{. } A \cap B = B \cap A: \text{ Commutative Law}.$$
$$2\text{. } A \cap \text{(} B \cap C \text{)} = \text{(} A \cap B \text{)} \cap C: \text{ Associative Law}.$$
$$3\text{. } A \cap \text{(} B \cap C \text{)} = \text{(} A \cap B \text{)} \cup \text{(} A \cap C \text{)}: \text{ Distributive Law}.$$
$$4\text{. } A \cap \text{(} A \cap B \text{)} = A: \text{ Absorption Law}.$$
$$5\text{. } A \cap U = U \cap A = A: \text{ Identity Law}.$$
$$6\text{. } A \cap A = A: \text{ Idempotent Law}.$$
$$7\text{. } (A^c)^c = A: \text{ Double Complement Law}.$$
$$8\text{. } A \cap \emptyset = \emptyset \cap A = \emptyset: \text{ Domination Law}.$$
$$9\text{. } A \cap A^c = \emptyset: \text{ Intersection with Complement Law}.$$
$$10\text{. } (A \cup B)^c = A^c \cap B^c: \text{ De Moirve's Law}.$$
The intersection can be swapped with the union to form another law (like, $A \cup B = B \cup A$ swapped as $A \cap B = B \cap A.$). Similarily, $U$ should be swapped with $\emptyset$ and vice versa.
\newline
[examples in word]
\newpage
\subsection{Generalised Set Operations}
Unions and Intersections; A saga:
$$1 \text{. } \cup_{i=1}^{n}A_i = A_1 \cup A_2 \cup \dots \cap A_n,$$
$$2 \text{. } \cap_{i=1}^{n} A_i = A_1 \cap A_2 \cap \dots \cap A_n.$$
Example:
\newline
\newline
$$A_k = {k, k + 1};$$
$$= \cup_{i=1}^{3}A_k = A_1 (\{1, 2\}) \cup A_2 (\{2, 3\}) \cup A_3 (\{3, 4\}),$$
$$= \{1, 2, 3, 4\}.$$
[example in word]
\newpage
\subsection{Russel's Paradox}
A set may contain another set as one of its elements.
\newline
This raises the possibility that a set may contain itself as an element.
\newline
\newline
\textbf{Problem: Try to let S be the set of all sets that are not elements of themselves, i.e.,}
$\mathbf{S = \{ A | A }\textbf{ is a set and } \mathbf{A \not\in A \}}.$
\newline
\textbf{Is S an element of itself?}
\newline
i)	If $S \in S$, then the definition of S implies that $S \not \in S$, a contradiction.
\newline
ii)	If $S \not \in S$, then the definition of S implies that $S \in S$, also a contradiction.
Hence neither $S \in S$ nor $S \not \in S$. This is Russell’s paradox.
\newpage
\subsection{Cartesian Product}
[example in word]
\newline
The Cartesian product of two sets A and B, denoted by $A \times B$, is the set of all ordered pairs from A to B:
\newline
\newline
\boxed{A \times B = \{(a,b)|a \in A \text{ and } b \in B\}}
\newline
\newline
If $|A| = m$ and $|B| = n$, then we have $|A \times B| = mn$.
\newline
\newline
Sets with more than 2 elements:
\newline
\newline
\textbf{Example:} $A = \{ a, b \}, B = \{ 1, 2, 3 \}.$
$$\text{Cartesian Product }(A \times B) = \{(a, 1), (a, 2), (a, 3), (b, 1), (b, 2), (b, 3)\}$$
(all of the ordered pairs -- combinations)
\newline
\newline
[example in word]
\newline
\newline
When X and Y are small finite sets, we can use an arrow diagram to represent a subset S of $X \times Y$ : we list the elements of X and the elements of Y , and then we draw an arrow from x to y for each pair $(x,y) \in S$.
\newpage
\subsection{Functions}
Example: Take 2 sets $X$ and $Y$, for which we have to find a function.
$$X = \{\text{all MATH 1081 students} \}, Y = \{ 0, 1, \dots , 84, 85, \dots, 100\}.$$
X: number of students; Y: marks from $0 - 100$.
\newline
Take function $f: X \to Y; \text{where } X \text{ is the domain and } Y \text{ is the co domain}.$
\newline
Ie, $f(x)$ = X's mark (Y).
\newline
Function $f: X \to Y$ satisfies $\{ (x, f(x)) | x \in X\} \subseteq X \times Y \text{ so that, for each } x \in X$;
\newline
1. $f(x)$ exists
\newline
2. $f(x)$ is unique
\newline
\newline
[example in word]
\newline
\newline
\boxed{\text{Note: be vary of the one-to-one function property lol}}
\newline
\newline
\underline{ Floor function and ceiling functions:}
\newline
\newline
1. Floor function (rounds down; smallest integer):
\newline
\newline
\boxed{\lfloor x \rfloor = \text{ max } \{ z \in Z | z \leq x\}.}
\newline
\newline
2. Ceiling function (rounds up; largest integer):
\newline
\newline
\boxed{\lceil x \rceil = \text{ min } \{ z \in Z | z \geq x\}.}
\newline
\newline
[example in word]
Domain/codomain: $\lfloor x \rfloor / \lceil x \rceil: \mathbb{R} \to \mathbb{Z}.$
\newline
\newline
Range($\lceil x \rceil$) = $\mathbb{Z}.$
\newline
\newline
[example in word]
\newpage
\subsection{Image and Inverse Image}
- The image of a set $A \subseteq X$ under a function $f : X \rightarrow Y \text{ is } f(A) = \{ y \in Y \text{ | } y = f(x) \text{ for some } x \in A\} = \{f(x)|x \in A\}$.
\newline
\newline
- The inverse image of a set $B \subseteq Y$ under a function $f : X \rightarrow Y is f^{-1}(B) = \{ x \in X |f(x) \in B \}$.
\newline
\newline
(image is just function values in the doman and inverse image is function values in range).
\newline
\newline
\boxed{\text{note: this is just function and inverse functions.}}
\newline
\newline
[example in word]
\newpage
\subsection{Injective, Surjective, Bijective}
Formal Definitions:
\newline
\newline
Recall that if f is a function from X to Y , then   for every $x \in X$, there is exactly one $y \in Y$ such that $f(x) = y$.
\newline
\newline
  1. We say that a function $f : X \rightarrow Y$ is injective or one-to-one if, for every $y \in Y$ , there is at most one $x \in X$ such that $f(x) = y$. 
  \newline
  Example: for all $x_1,x_2 \in X$, if $f(x_1) = f(x_2)$ then $x_1 = x_2$.  
  \newline
  \newline
  2. We say that a function $f : X \rightarrow Y$ is surjective or onto if, for every $y \in Y$ , there is at least one $x \in X$ such that $f(x) = y$.
  the range of f is the same as the codomain of f (range(f) = Y).
  \newline
  \newline
  3. We say that a function $f : X \rightarrow Y$ is bijective if $f$ is both injective and surjective (one-to-one and onto).   
  \newline
  \newline
  for every $y \in Y$ , there is exactly one $x \in X$ such that $f(x) = y$.
  \newline
  [example in word]
  \newpage
  \subsection{Composition of Functions}
    For functions $f : X \rightarrow Y$ and $g : Y \rightarrow Z$, the composite of f and g is the function $g \circ f : X \rightarrow Z$ defined by $(g \circ f)(x) = g(f(x))$ for all $x \in X$.
    \newline
    \newline
    The composite function $g \circ f$ exists whenever the range of f is a subset of the domain of g.
    \newline
    \newline
    In general, $g \circ f$ and $f \circ g$ are not the same composite functions. Associativity of composition (assuming they exist): $h \circ (g \circ f) = (h \circ g) \circ f$.  
\newline
\newline
\textbf{ Example: Take sets $X = \{\text{ all MATH1081 students }\}, Y = \{ 0, 1, \dots, 100 \}, Z = \{F, P, CR, D, HD\}.$  }
\newline
\newline 
Maps: $f = X \rightarrow Y; g = Y \rightarrow Z.$
\newline
\newline
A) $g \circ f: X \rightarrow Z.$
\newline
$(f \circ g) (y) = f(g(y)).$
\newline
[examples in word]
\newpage
\subsection{Identity and Inverse Functions}
\underline{ Identity Function:}
\newline
\newline
\boxed{i_x: x \rightarrow x; i_x(x) = x.}
\newline
\newline
For any function $f : X \rightarrow Y$ , we have $f \circ i_x = f = i_y \circ f$.   A function $g : Y \rightarrow X$ is an inverse of $f : X \rightarrow Y$ if
$g(f(x)) = x$ for all $x \in X$ and $f(g(y)) = y$ for all $y \in Y$,
\newline
or equivalently, $g \circ f = i_x$ and $f \circ g = i_y$ .
\newline
\newline
\boxed{\text{1. A function can have at most one inverse.}}
\newline
\newline
If $f : X \rightarrow Y$ has an inverse, then we say that $f$ is invertible, and we denote the inverse off by $f^{-1}$. Thus,$f^{-1} \circ f = i_x$ and $f \circ f^{-1} = i_y$.
\newline
\newline
If $g$ is the inverse of $f$, then $f$ is the inverse of $g$. Thus, $(f^{-1})^{-1} = f$.
\newline
\newline
[example in word]
\newpage
\subsection{Inverse Function Proofs}
\underline{ Theorem and Proof: }
\newline
\newline
\boxed{1. \text{ A function } f: X \rightarrow Y \text{ has at most 1 inverse} }
\newline
\newline
Proof:
\newline
\newline
$$\text{ Let } g_1, g_2: Y \rightarrow X \text{ be inverse of }f.$$
$$\text{ Then }: g_1 = g_i \circ i_y$$
$$ = g_i \circ (f \circ g_2)$$
$$ = (g_i \circ f) \circ g_2$$
$$ = i_x \circ g_2$$
$$ = g_2 \text{ End of proof }.$$
[example in word]
\newpage
\section{Number Theory and Relations}
\subsection{Numbers and Divisibility}
[topic 2 done in word (SteelsSlides1): maybe put in definitions here ?? that depends]
\newline
\newline
Number Set Notation:
\newline
\newline
1. The positive integers: $\mathbb{Z}^+ = \{ 1, 2, 3, \dots\},$
\newline
2. The natural numbers: $\mathbb{N} = \{ 0, 1, 2, 3, \dots\},$
\newline
3. The integers: $\mathbb{Z} = \{\dots, -2, -1, 0, 1, 2, \dots\},$
\newline
4. The rational numbers: $\mathbb{Q} = \{ m/n : m \in \mathbb{Z}, n \in \mathbb{Z}^+ \},$
\newline
5. The real numbers, $\mathbb{R}$ and the complex numbers $\mathbb{C}.$
\newline
\newline
Tests (Divisibility):
\newline
\newline
1. $2\text{ | } N$ if and only if the decimal expansion of N ends in an even integer
\newline
2. $5 \text{ | } N$ if and only if the last decimal digit of N is 5 or 0.
\newline
3. $3 \text{ | } N$ if and only if the sum of the decimal digits of N is divisible by 3.
\newline
3': $9 \text{ | } N$ if and only if the sum of the decimal digits of N is divisible by 9.
\newline
4. $11 \text{ | } N$if the alternating sum of the decimal digits of N is divisible by 11. 
\newline
(example: $1232 = 1 - 2 + 3 - 2 = 0$)
\newline
\newline
[proof in word]
\newpage
\subsection{Primes}
[in word]
\newline
\newline
\underline{Primes Definition: Formal:}
Another way of saying this is if $p$ is prime:
$$x \text{ | } p	\text{ implies }	x \in \{ -1,1,-p,p \}$$.
\newline
\newline
\underline{Theorems:}
\newline
\newline
1. If p is prime and $p | ab$, then $p | a$ or $p | b$,
\newline
2. If n is composite, then it has a prime factor less than or equal to $\sqrt[2]{n}$,
\newline
3. If no prime less than or equal to $\sqrt[2]{n}$ divides $n$ then $n$ is a prime,
\newline
4. Every integer $n \geq 2$ can be written uniquely as a product of a finite number of primes in increasing order i.e. $n = p_1^{m_1} * p_2^{m_2} \dots p_k^{m_k}$
for primes $p_1 < p_2 < \dots < p_k$ and exponents $m_1,m_2, \dots ,m_k \in \mathbb{Z}^+$.
\newline
\newline
Open Results about Primes:
\newline
\newline
1. A prime of the form $2^n + 1$ is called a Fermat prime.
\newline
2. A prime of the form $2^{n} - 1$ is called a Mersenne prime.
\newline
3. Two primes that differ by 2, are called twin primes.
For example, $3$ and $5$ are twin primes; so are $29$ and $31$.
\newline
4. The Goldbach Conjecture is that they are: it has been proved true for all numbers with fewer than about $17$ digits.
\newpage
\subsection{Common Divisors and Multiples}
[mostly on word]
\newline
All $a,b \in \mathbb{Z}$ have (at least) one common divisor, namely $1$, and so we can define the following:
\newline
For $a,b \in \mathbb{Z}$, not both zero, the positive integer d such that
$$1.	d \text{ | } a and d \text{ | } b,$$
$$2.	If c \text{ | } a and c \text{ | } b then c \le d.$$
is called the greatest common divisor of $a$ and $b$.
We write $d = gcd(a,b)$.
\newline
\newline
\boxed{ \text{ Begin by writing $a$ and $b$ as a product of primes. }}
\newline
\newline
\underline{ Properties of GCD:}
\newline
\newline
1. $\text{gcd }(a,b)$ is not affected by the signs of $a$ or $b$
\newline
2. Condition $(2)$ in the definition of gcd can be replaced by $(2')$ if $c \text{ | } a$ and $c \text{ | } b$ then $c \text{ | } d$.
\newline
3. For $a \in \mathbb{Z}^+, \text{gcd }(a,0) = a$.
\newline
\newline
\underline{Least Common Multiple}
\newline
\newline
All $a,b \in \mathbb{Z}$ have (at least) one common multiple, namely $ab$, and so we can define the following:
For $a,b \in \mathbb{Z}$, not both zero, the positive integer l such that
$$1)	a \text{ | } l \text{ and } b \text{ | } l$$
$$2)	\text{If } a \text{ | } c \text{ and } b \text{ | } c \text{ then } l \le c \text{ is called the least common multiple of } a \text{ and } b.$$ 
\newline
We write $l = \text{ lcm }(a,b)$.
\newline
\newline
Theorem:
\newline
\newline
\boxed{\text{For all positive integers $a$ and $b$; gcd}(a,b) x \text{ lcm }(a,b) = ab.}
\newpage
\subsection*{Quotient and Remainder}
\
[mostly in word]
\newline
\newline
\underline{The Quotient-Remainder Theorem (aka The Division Algorithm)}
\newline
\newline
If $a \in \mathbb{Z}$ and $b \in \mathbb{Z}^+$, then there exist unique $q,r \in Z$ such that (q: quotient; r: remainder):
$$a = bq + r	\text{ and }	0 \le r < b.$$
\newline
\newline
\boxed{\text{ Note: $q$ can be found using floor function; $q = \lfloor a / b \rfloor.$; then $r = a - qb$.}}
\newpage
\subsection{The Euclidean Algorithm}
\
[mostly in word]
\newline
\newline
\boxed{\text{If $a = bq + r$ then gcd$(a,b) =$ gcd$(b,r)$.}}
\newline
\newline
\underline{The Euclidian Algorithm: General Case [steps]}
\newline
\newline
$1)$	Let $a$ and $b$ be integers with $a > b \geq 0.$
\newline
$2)$	If $b = 0$, then gcd$(a,b) = a.$
\newline
$3)$	If $b > 0$, use the Quotient-Remainder theorem to write $a = bq + r$ where $0 \leq r < b$.	Then by our prevous result, gcd$(a,b) =$ gcd$(b,r).$
\newline
$4)$	Repeat steps 2 and 3 to find gcd$(b,r)$.
\newline
\newline
Example: Find gcd$(708, 540)$
$$708 = 540 \cdot 1 + 168,$$
$$540 = 168 \cdot 3 + 36,$$
$$168 = 36 \cdot 4 + 24,$$
$$36 = 24 \cdot 1 + 12,$$ 
$$24 = 12 \cdot 2 + 0.$$
So,
$$\text{gcd }(708,540) = 12.$$
\boxed{\text{Note: gcd is the last non-zero remainder.}}
\newline
\newline
\underline{Bezout's Identity}
\newline
\newline
For $a,b \in Z$ not both zero, there exist integers $x$ and $y$ (not unique) such that:
\newline
\newline
\boxed{ \text{gcd }(a,b) = ax + by.}
\newline
\newline
\boxed{\text{Theorem: Integers $a$ and $b$ are relatively prime if and only if there exists $x,y \in Z$ such that $ax + by = 1$.}}
\newline
\newline
\underline{Extended Euclidean Theorem:}
The Extended Euclidean Algorithm is a more efficient way of finding the numbers in Bézout's Identity: In looking for gcd$(a,b)$, assume $a > b > 0$.
\newline
1. We make up a table with five columns labelled $i, q_i, r_i, x_i, y_i$, where $i$ labels the rows.
\newline
2. We set row $1$ to be $1,0,a,1,0$ and row $2$ to be $2,0,b,0,1.$
Thus $q_1 = q_2 = 0; r_1 = a, r_2 = b; x_1 = y_2 = 1; x_2 = y_1 = 0.$
\newline
3. Then for $i$ from $3$ onwards, $q_i$ is the quotient on dividing $r_{i-2}$ by $r_{i-1}$ ($a$ divided by $b$ in the first case).
\newline
4. Then subtract $q_i$ times the rest of row $i - 1$ from row $i - 2$.
\newline
5. Repeat until we get $r_{n+1} = 0$ for some $n$, then stop.
Then the gcd is $r_n$ and $r_n = ax_n + by_n$, that is the last row before $r_i$ was zero gives the gcd, the $x$ and the $y$.
\newline
In fact a similar identity holds at each step: $r_i = ax_i + by_i$.
\newpage
\subsection{Modular Arithmetic}
[mostly in word]
\newline
\newline
Let $m \geq 2$ be an integer. We say that a and b are congruent modulo $m$ if $m | (a - b)$.
\newline
We write this as:
$$a \cong b (mod m).$$
The reason we have taken our modulus m to be greater than $2$ is that
\newline
1)	As $m | (a - b)$ iff $- m | (a - b)$, there is nothing to be gained from using negative moduli.
\newline
2)	All numbers are congruent modulo $1$, so that is not interesting.
\newline
3)	divisibility by $0$ is not defined.
\newline
\newline
\underline{Theorem}
\newline
For integers $a, b$and $m, a \cong b (mod m)$ if and only if there is an integer $k$ such that $a = b + km$.
\newline
\newline
\underline{Arithmetic with Congrences}
\newline
Suppose $a \cong b (mod m)$ and $c \cong d (mod m)$.
\newline
Then
$$(1a) (a + c) \cong (b + d) (mod m).$$
$$(1b) (a - c) \cong (b - d) (mod m).$$
$$(2)	ac \cong bd (mod m).$$
$$(3)	an \cong bn (mod m) for all n \in N.$$
$$(4)	If k \text{ | } m then a \cong b (mod k).$$
\boxed{note: never divide congruences}
\newline
\newline
Applications of Congruence Arithmetic:
\newline
1. Pseudo-random Numbers
\newline
2. Equations with no solutions
\newpage
\section{Logic and Proofs}
\subsection{Introduction to Logic and Proofs}
Mathematical proof: consists of logical deduction on the basis of agreed premises. Apart from human error the results are certain.
\newline
\newline
\underline{ Techniques for Proof: }
\newline
\newline
•	Always explain what you are doing, and your reasons for drawing your conclusions.
\newline
•	Simplify!
\newline
•	Keep the aim in mind.
\newline
•	Plan a solution. 
\newline
• Work on one side of an equation or inequation to relate it to the other.
\newline
\newline
To study proofs: you always have to practice!! No matter what. There are techniques,
but most of the time you will have to practice proofs.
\newline
What we study is methods of proof and logic.
\newpage
\subsection*{Example of Proofs}
[in word?? mostly?? i think so yeah]
\newline
\newline
Example: $\frac{1}{1000} - \frac{1}{1001} < \frac{1}{1000000}.$
\newline
\underline{Techniques:} 
\newline
1. Common denominator (simple answer); 
\newline
2. Reciprocals; 1/2a < 1/a kinda way (bigger denominator smaller fractions)
\newline
3. Simplify
\newline
\newline
\textbf{Proof}. We have
$$\frac{1}{1000} - \frac{1}{1001} = \frac{1001 - 1000}{1000 x 1001} = \frac{1}{1001000}$$
But $1001000 > 1000000$, and both are positive numbers, so
$$\frac{1}{1001000} < \frac{1}{1000000}$$  
Therefore
$$\frac{1}{1000} - \frac{1}{1001} < \frac{1}{1000000}.$$  
\newline
\boxed{\text{It is handy to use calculators, but this is generally bad practice as there is no understanding.}}
\newline
\newline
Equality proofs should ideally not be proved by calculators!
\end{document}