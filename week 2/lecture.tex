\documentclass{article}
\usepackage{amsmath,amsthm,amsfonts}                        % AMS Math
\usepackage{thmtools}                                       % Theorem Tools
\usepackage{bm}        
\usepackage{tabto}                                          % Bold Math

\title{MATH1081 notes}
\author{Nira (z5417727)}
\date{September 17th, 2022}
\begin{document}
\maketitle 
\tableofcontents
\newpage
\section{Topic 1}
\subsection{Introduction}
1. addition, multiplication, division and subtraction
\newline
2. Mainly dealing with finite sets

\newpage
\subsection{Sets and subsets}
\boxed{ $\text{A set is a well defined collection of distinct objects}$ }
\newline
\newline
Example: $S = \{1, a, 3\}, A =\{\Pi, 1\}.$
\newline
1. $e \notin A$; it is not in A
\newline
2. For example, if A is a set of all integers; $\{ \text{all even integers} \}$
    = $\{n \in \mathbb{R} | \text{n is even}\}$.
\newline
3. We can remove superfluos items (elements that occur more than one).
$A = \{ 1, 2, 3, 3 \}$ where $3$ can be removed.
\newline
\newline
\newline
Example:
\newline
$A = \{ 1, 2, 3 \},
B = \{2, 3, 1 \}, 
C = \{1, 2, 3, 3 \},
D = \{ 1 , 3 \}.$
\newline
Here, D is a proper subset of A, B, C; A, B, C are supersets of D.
\newline
\boxed {$$\subseteq: \text{Subset (proper subset)}, $$
$$ \supseteq: \text{Superset}.$$}
\newline
\newline
1. To prove if a set is a proper subset; do the following:
\newline
For example, if $D \in A$, then check if $e \in D$
\newline
If $e \in D,$ then $e \in A$.
Thus, it would be a proper subset (here, e is just an element).
\newline
\newline
2. To prove that two sets are equal;
\newline
For example, if A = B, prove:
\newline
i) $A \subseteq B$; if an element is in A, then the element is in B.
\newline
ii) $B \subseteq A$; if an element is in B, then the element is in A.
\newpage
\subsection{Power Sets and Stability}
Subsets of $A = \{1, 2, 3\} $:
\newline
1. Could throw everything out to get empty set $\Phi$,
\newline
2. One element each: $\{ 1 \}, \{ 2 \}, \{ 3 \}$,
\newline
3. Two elements: $\{ 1, 2 \}, \{ 2, 3 \}, \{ 1, 3 \}$,
\newline
4. Set itself: $A$.
\newline
The set containing 1, 2, 3, 4 is called the powerset of A.
\newline
\newline
Given $
A = \{1, 2, 3\}, B = \{1, 2, 3, 3\}, C = \{1, 3\}, D = \{1, 3\}$, where \newline
$A = B$, $C \subseteq A, B$ and $D \not \subseteq A, B, C$.
\newline
1. size of A = 3, B = 3, C = 2, D = 2.
\newline
[Exercise with A  = {0, 1, {0, 1}}, B done in word].
\newpage
\subsection{Set Operations}
Boolean Operators ("not" operation in programming): 
\newline
1. Complement:
\newline
Let there be a set A in U (A: all of the people in the video, U: universal set of everyone in the world, $A^c =$ complement of A).
\newline
\newline
\boxed { $$A^c = \{ x \in U | x \not \in A \}.$$}
\newline
\newline
2. Intersecting ("and" operation in programming):
\newline
If there is $A, B$, intersecting, 
\newline
\newline
\boxed{ $$A \cap B = \{ x \in A | x \in B\}.$$}
\newline
\newline
3. Union ("or" operation in programming):
If there is $A, B$, A or B is:
\newline
\newline
\boxed{ $$A \cup B = \{ x \in U | x \in A \text { or } x \in B\}.$$}
\newline
\newline
4. Difference:
If there is $A, B$, intersecting, 
\newline
\newline
\boxed{ $$A - B = \{ x \in A | x \in B \}.$$}
\newline
\newline
[examples in word doc]
\newpage
\subsection{The Inclusion-Exclusion Principle}
[example in Word]
\newline
\boxed{ $$|A \cup B| = |A| + |B| - | A \cap B|$$.}
\newline
\newline
For three elements, 
\newline
\newline
\boxed{ $$|A \cup B \cup C| = |A| + |B| + |C| - | A \cap B| - |A \cap C| - |B \cap C| + |A \cap B \cap C|$$.}
\newline
\newline
[example in word]
\newpage
\subsection{Sets Proofs}
[proof question in word]
\newline
Hints for proofs:
\newline
1. To prove that $S \subseteq T$, we can assume that $x \in S$ and show that $x \in T$.
\newline
2. 
To prove that $S = T$, we can show that $S \subseteq T$ and $T \subseteq S$.
\newline
\newline
Scaffold: 
\newline
\boxed{ \text{Proof: Suppose that \dots 
\newline
\dots (proof)
\newline
we see that/ it follows \dots (conclusion) (end with shaded box to indicate end of proof.)}}
\newline
\newline
Note that the "Suppose that" part of the proof is usually whatever the if statment mentions.
\newline
\newline
For example, if the question is "Prove that if $A \cap B = A$, then $A \cup B = B$, then the proof starts like this:
\newline
\boxed{\underline{Proof}: \text{ Suppose that } A \cap B = A. }
\newline
For questions like "is this statement true", there are two ways to approach the question:
\newline
1. If the statement is true (if you think it is true), then prove it.
\newline
2. If the statement is false, then give a counter-example that proves it false. 
\newline
[examples in word]
\newpage
\subsection{Laws of Set Algebra}
\underline{ Laws of Set Algebra}
\newline
$$1\text{. } A \cap B = B \cap A: \text{ Commutative Law}.$$
$$2\text{. } A \cap \text{(} B \cap C \text{)} = \text{(} A \cap B \text{)} \cap C: \text{ Associative Law}.$$
$$3\text{. } A \cap \text{(} B \cap C \text{)} = \text{(} A \cap B \text{)} \cup \text{(} A \cap C \text{)}: \text{ Distributive Law}.$$
$$4\text{. } A \cap \text{(} A \cap B \text{)} = A: \text{ Absorption Law}.$$
$$5\text{. } A \cap U = U \cap A = A: \text{ Identity Law}.$$
$$6\text{. } A \cap A = A: \text{ Idempotent Law}.$$
$$7\text{. } (A^c)^c = A: \text{ Double Complement Law}.$$
$$8\text{. } A \cap \emptyset = \emptyset \cap A = \emptyset: \text{ Domination Law}.$$
$$9\text{. } A \cap A^c = \emptyset: \text{ Intersection with Complement Law}.$$
$$10\text{. } (A \cup B)^c = A^c \cap B^c: \text{ De Moirve's Law}.$$
The intersection can be swapped with the union to form another law (like, $A \cup B = B \cup A$ swapped as $A \cap B = B \cap A.$). Similarily, $U$ should be swapped with $\emptyset$ and vice versa.
\newline
[examples in word]
\newpage
\subsection{Generalised Set Operations}
Unions and Intersections; A saga:
$$1 \text{. } \cup_{i=1}^{n}A_i = A_1 \cup A_2 \cup \dots \cap A_n,$$
$$2 \text{. } \cap_{i=1}^{n} A_i = A_1 \cap A_2 \cap \dots \cap A_n.$$
Example:
\newline
\newline
$$A_k = {k, k + 1};$$
$$= \cup_{i=1}^{3}A_k = A_1 (\{1, 2\}) \cup A_2 (\{2, 3\}) \cup A_3 (\{3, 4\}),$$
$$= \{1, 2, 3, 4\}.$$
[example in word]
\newpage
\subsection{Russel's Paradox}
A set may contain another set as one of its elements.
\newline
This raises the possibility that a set may contain itself as an element.
\newline
\newline
\textbf{Problem: Try to let S be the set of all sets that are not elements of themselves, i.e.,}
$\mathbf{S = \{ A | A }\textbf{ is a set and } \mathbf{A \not\in A \}}.$
\newline
\textbf{Is S an element of itself?}
\newline
i)	If $S \in S$, then the definition of S implies that $S \not \in S$, a contradiction.
\newline
ii)	If $S \not \in S$, then the definition of S implies that $S \in S$, also a contradiction.
Hence neither $S \in S$ nor $S \not \in S$. This is Russell’s paradox.
\newpage
\subsection{Cartesian Product}
[example in word]
\newline
The Cartesian product of two sets A and B, denoted by $A \times B$, is the set of all ordered pairs from A to B:
\newline
\newline
\boxed{A \times B = \{(a,b)|a \in A \text{ and } b \in B\}}
\newline
\newline
If $|A| = m$ and $|B| = n$, then we have $|A \times B| = mn$.
\newline
\newline
Sets with more than 2 elements:
\newline
\newline
\textbf{Example:} $A = \{ a, b \}, B = \{ 1, 2, 3 \}.$
$$\text{Cartesian Product }(A \times B) = \{(a, 1), (a, 2), (a, 3), (b, 1), (b, 2), (b, 3)\}$$
(all of the ordered pairs -- combinations)
\newline
\newline
[example in word]
\newline
\newline
When X and Y are small finite sets, we can use an arrow diagram to represent a subset S of $X \times Y$ : we list the elements of X and the elements of Y , and then we draw an arrow from x to y for each pair $(x,y) \in S$.
\newpage
\subsection{Functions}
Example: Take 2 sets $X$ and $Y$, for which we have to find a function.
$$X = \{\text{all MATH 1081 students} \}, Y = \{ 0, 1, \dots , 84, 85, \dots, 100\}.$$
X: number of students; Y: marks from $0 - 100$.
\newline
Take function $f: X \to Y; \text{where } X \text{ is the domain and } Y \text{ is the co domain}.$
\newline
Ie, $f(x)$ = X's mark (Y).
\newline
Function $f: X \to Y$ satisfies $\{ (x, f(x)) | x \in X\} \subseteq X \times Y \text{ so that, for each } x \in X$;
\newline
1. $f(x)$ exists
\newline
2. $f(x)$ is unique
\newline
\newline
[example in word]
\newline
\newline
\boxed{\text{Note: be vary of the one-to-one function property lol}}
\newline
\newline
\underline{ Floor function and ceiling functions:}
\newline
\newline
1. Floor function (rounds down; smallest integer):
\newline
\newline
\boxed{\lfloor x \rfloor = \text{ max } \{ z \in Z | z \leq x\}.}
\newline
\newline
2. Ceiling function (rounds up; largest integer):
\newline
\newline
\boxed{\lceil x \rceil = \text{ min } \{ z \in Z | z \geq x\}.}
\newline
\newline
[example in word]
Domain/codomain: $\lfloor x \rfloor / \lceil x \rceil: \mathbb{R} \to \mathbb{Z}.$
\newline
\newline
Range($\lceil x \rceil$) = $\mathbb{Z}.$
\newline
\newline
[example in word]
\newpage
\subsection{Image and Inverse Image}
- The image of a set $A \subseteq X$ under a function $f : X \rightarrow Y \text{ is } f(A) = \{ y \in Y \text{ | } y = f(x) \text{ for some } x \in A\} = \{f(x)|x \in A\}$.
\newline
\newline
- The inverse image of a set $B \subseteq Y$ under a function $f : X \rightarrow Y is f^{-1}(B) = \{ x \in X |f(x) \in B \}$.
\newline
\newline
(image is just function values in the doman and inverse image is function values in range).
\newline
\newline
\boxed{\text{note: this is just function and inverse functions.}}
\newline
\newline
[example in word]
\newpage
\subsection{Injective, Surjective, Bijective}
Formal Definitions:
\newline
\newline
Recall that if f is a function from X to Y , then   for every $x \in X$, there is exactly one $y \in Y$ such that $f(x) = y$.
\newline
\newline
  1. We say that a function $f : X \rightarrow Y$ is injective or one-to-one if, for every $y \in Y$ , there is at most one $x \in X$ such that $f(x) = y$. 
  \newline
  Example: for all $x_1,x_2 \in X$, if $f(x_1) = f(x_2)$ then $x_1 = x_2$.  
  \newline
  \newline
  2. We say that a function $f : X \rightarrow Y$ is surjective or onto if, for every $y \in Y$ , there is at least one $x \in X$ such that $f(x) = y$.
  the range of f is the same as the codomain of f (range(f) = Y).
  \newline
  \newline
  3. We say that a function $f : X \rightarrow Y$ is bijective if $f$ is both injective and surjective (one-to-one and onto).   
  \newline
  \newline
  for every $y \in Y$ , there is exactly one $x \in X$ such that $f(x) = y$.
  \newline
  [example in word]
  \newpage
  \subsection{Composition of Functions}
    For functions $f : X \rightarrow Y$ and $g : Y \rightarrow Z$, the composite of f and g is the function $g \circ f : X \rightarrow Z$ defined by $(g \circ f)(x) = g(f(x))$ for all $x \in X$.
    \newline
    \newline
    The composite function $g \circ f$ exists whenever the range of f is a subset of the domain of g.
    \newline
    \newline
    In general, $g \circ f$ and $f \circ g$ are not the same composite functions. Associativity of composition (assuming they exist): $h \circ (g \circ f) = (h \circ g) \circ f$.  
\newline
\newline
\textbf{ Example: Take sets $X = \{\text{ all MATH1081 students }\}, Y = \{ 0, 1, \dots, 100 \}, Z = \{F, P, CR, D, HD\}.$  }
\newline
\newline 
Maps: $f = X \rightarrow Y; g = Y \rightarrow Z.$
\newline
\newline
A) $g \circ f: X \rightarrow Z.$
\newline
$(f \circ g) (y) = f(g(y)).$
\newline
[examples in word]
\newpage
\subsection{Identity and Inverse Functions}
\underline{ Identity Function:}
\newline
\newline
\boxed{i_x: x \rightarrow x; i_x(x) = x.}
\newline
\newline
For any function $f : X \rightarrow Y$ , we have $f \circ i_x = f = i_y \circ f$.   A function $g : Y \rightarrow X$ is an inverse of $f : X \rightarrow Y$ if
$g(f(x)) = x$ for all $x \in X$ and $f(g(y)) = y$ for all $y \in Y$,
\newline
or equivalently, $g \circ f = i_x$ and $f \circ g = i_y$ .
\newline
\newline
\boxed{\text{1. A function can have at most one inverse.}}
\newline
\newline
If $f : X \rightarrow Y$ has an inverse, then we say that $f$ is invertible, and we denote the inverse off by $f^{-1}$. Thus,$f^{-1} \circ f = i_x$ and $f \circ f^{-1} = i_y$.
If $g$ is the inverse of $f$, then $f$ is the inverse of $g$. Thus, $(f^{-1})^{-1} = f$.
\end{document}