\documentclass{article}
\usepackage[english]{babel}
\usepackage[letterpaper,top=2cm,bottom=2cm,left=3cm,right=3cm,marginparwidth=1.75cm]{geometry}
\usepackage{amsmath}
\usepackage{amssymb}
\usepackage{graphicx}
\usepackage[colorlinks=true, allcolors=black]{hyperref}

\title{Maths Assignment - 1081}
\author{z5417727}

\begin{document}
\maketitle
{\LARGE \tableofcontents}
\newpage
\section{Question 1}
\subsection{a)}
\textbf{ Prove that in modulo $9$, it is not possible for a perfect square to be congruent to $2, 3, 5, 6 \text{ or } 8$.}
\newline
\newline
\textbf{ Proposition: } For any integer $n \in \mathbb{Z}$, we say that $n^2 \equiv 0, 1, 4, 7 (\text{ mod } 9).$ 
\newline
\newline
\textbf{Proof: } This can be deduced by finding the squares of $0, 1, 2, 3, 4$ respectively.
$$0^2 \equiv 0 (\text{mod }9),$$
$$1^2 \equiv 1 (\text{mod }9),$$
$$2^2 \equiv 4 (\text{mod }9),$$
$$3^2 \equiv 0 (\text{mod }9),$$
$$4^2 \equiv 7 (\text{mod }9).$$
Through finding the modulo of $9$, we find the similar rule applied to $5$ through $8$ (since $9^2 \equiv 0 (\text{ mod } 9)$).
$$5^2 \equiv (-4)^2 \equiv 7 (\text{mod }9),$$
$$6^2 \equiv (-3)^2 \equiv 0 (\text{mod }9),$$
$$7^2 \equiv (-2)^2 \equiv 4 (\text{mod }9),$$
$$8^2 \equiv (-1)^2 \equiv 1 (\text{mod }9).$$
Here, we see that the modulo of perfect squares always end with the digits $0, 1, 4$ and $7$. Thus, it can be proved that in modulo $9$, it is not possible for a perfect square to be congruent to $2, 3, 5, 6, \text{ or } 8$.
\newpage
\subsection{b)}
\textbf{ Hence (and not otherwise) prove that there do not exist three consecutive integer values of $n$ for which $41n + 39$ is a perfect square. }
\newline
\newline
Consider a number $n - 1$, $n$ and $n + 1$ for $n \in \mathbb{Z}$. 
Then, we see that the numbers are: 
$$41(n - 1) + 39, 41(n) + 39, 41(n + 1) + 39.$$
\newline
\newline
\textbf{ Proposition } For $41(n - 1) + 39, 41(n) + 39 \text{ and } 41(n + 1) + 39$ to be perfect squares, they should not be congruent to $2, 3, 5, 6 \text{ or } 8$ in modulo $9$ (this is proved in q$1$ (a)).
\newline
\newline
\textbf{Proof } Consider $41n + 39$ as a perfect square.
\newline
\newline
$$41n + 39 \text{ as a perfect square } \Rightarrow 41n + 39 = k^2, \text{ where } k \in \mathbb{Z}.$$
Here, we can use the proof from q1 (a) to deduce that $k^2 \text{ mod } 9$ would give $0, 1, 4$ or $7$ as the remainder since it is a perfect square. 
\newline
\newline
However, when we check the number $41(n - 1) + 39,$
$$\Rightarrow 41n - 41 + 39,$$
$$\Rightarrow (41n + 39) - 41,$$
$$\Rightarrow k^2 - 41.$$
Thus, we can consider the modulo of $9$ for $k^2 + 41$:
$$\Rightarrow (k^2 - 41) (\text{mod } 9),$$
$$\Rightarrow (k^2 (\text{mod } 9) - 41 (\text{mod } 9)) (\text{mod } 9). (\text{modular subtraction})$$
Here, we know that $41 \equiv 5 (\text{mod } 9),$ and $k^2$ gives a remainder of either $0, 1, 4, 7.$ Consider each of the cases individually:
\newline
$1) $ $k^2 \equiv 0 (\text{mod } 9)$:
\newline
\newline
$$\Rightarrow (k^2 (\text{mod } 9) - 41 (\text{mod } 9)) (\text{mod } 9).$$
$$\Rightarrow (0 - 5) (\text{mod } 9),$$
$$\Rightarrow -5 (\text{mod } 9),$$
$$\Rightarrow -5.$$
Since the $(k^2 - 41) \equiv -5 (\text{mod } 9),$ this means that it is not a perfect square (as proven in q1 a)).
\newline
$2) $ $k^2 \equiv 1 (\text{mod } 9)$:
\newline
\newline
$$\Rightarrow (k^2 (\text{mod } 9) + 41 (\text{mod } 9)) (\text{mod } 9).$$
$$\Rightarrow (1 - 5) (\text{mod } 9),$$
$$\Rightarrow -4 (\text{mod } 9),$$
$$\Rightarrow -4.$$
Since the $(k^2 - 41) \equiv -4 (\text{mod } 9),$ this means that it is not a perfect square (as proven in q1 a)).
\newline
$3) $ $k^2 \equiv 4 (\text{mod } 9)$:
\newline
\newline
$$\Rightarrow (k^2 (\text{mod } 9) - 41 (\text{mod } 9)) (\text{mod } 9).$$
$$\Rightarrow (4 - 5) (\text{mod } 9),$$
$$\Rightarrow -1 (\text{mod } 9),$$
$$\Rightarrow -1.$$
Since the $(k^2 - 41) \equiv 0 (\text{mod } 9),$ this means that it is not a perfect square (as proven in q1 a)).
\newline
$4) $ $k^2 \equiv 7 (\text{mod } 9)$:
\newline
\newline
$$\Rightarrow (k^2 (\text{mod } 9) - 41 (\text{mod } 9)) (\text{mod } 9).$$
$$\Rightarrow (7 - 5) (\text{mod } 9),$$
$$\Rightarrow 2 (\text{mod } 9),$$
$$\Rightarrow 2.$$
Since the $(k^2 - 41) \equiv 2 (\text{mod } 9),$ this means that it is not a perfect square (as proven in q1 a)).
\newline
We see that for each case, $41(n - 1) + 39$ can never be a perfect square if $41n + 39$ is a perfect square.
\newline
\newline
Therefore, we can say that there do not exist three consecutive integer values of $n$ for which $41n + 39$ is a perfect square.
\newpage
\section{Question 2}
\textbf{ A certain relation $\star$ is defined on the set $\mathbb{Z}^+$ by:
\newline
$$x \star y \text{ if and only if every factor of } x \text{ is a factor of } y.$$
For each of the questions below, be sure to provide a proof supporting your answer. }
\subsection{a)}
\textbf{ Is $\star $ reflexive? }
\newline
\newline
\textbf{Theorem: } If $\star$ is to be reflexive, then $x \sim x.$
\newline
For example, let $y = kx,$ where $k \in \mathbb{Z}^+.$
If we swap the $x$ and $y$ values, so we get $x = kx$. Now, since $x = kx$ is only true when $x = 1,$ we can conclude that $x \star y$ is not reflexive.
\newline
\newline
[not done yet]
\newpage
\subsection{b)}
\textbf{Is $\star $ symmetric?}
\newline
\newline
\textbf{Theorem: } If $\star$ is symmetric, then $x \sim y \leftrightarrow y \sim x.$
\newline
\newline
[not done yet]
\newpage
\subsection{c)}
\textbf{Is $\star $ anti-symmetric? }
\newline
\newline
\textbf{Theorem: } If a set $A \leq B, B \leq A \rightarrow A = B.$
\newline
\newline
[not done yet]
\newpage
\subsection{d)}
\textbf{ Is $\star $ transitive?}
\newline
\newline
If a set $A \leq B, B \leq C \rightarrow A \leq C.$
\newline
\newline
[not done yet]
\newpage
\subsection{e)}
\textbf{ Is $\star $ an equivalence relation, a partial order, both or neither?}
\newline
\newline
[not done yet]
\newpage
\section{Question 3}
\textbf{ Consider the two functions $f: X \rightarrow Y$ and $g: Y \rightarrow Z$ for non-empty sets $X, Y, Z$.
Decide whether each of the following statements is true or false, and prove each claim.}
\subsection{a)}
\textbf{ If $g \circ f$ is injective, then $g$ is injective.}
\newline
\newline
\textbf{ Counterexample }
\newline
\newline
Consider sets $X = \{1\}, Y = \{2, 3\}, Z = \{4\}.$
\newline
\newline
Function $g \circ f$ implies that $g \circ f: X \rightarrow Z$ (since $f: X \rightarrow Y \text{ and } g: Y \rightarrow Z$). Therefore, $g(f(1)) = 4.$
\newline
This makes it an injective function as it is one to one.
\newline
\newline
However, for the function $g$, $g(2) = g(3) = 4, $ making the function non-injective.
\newline
\newline
Therefore, by a counterexample, we can conclude that the statement "If $g \circ f$ is injective, then $g$ is injective" is false.
\newpage
\subsection{b)}
\textbf{ If $g \circ f$ is injective, then $f$ is injective. }
\newline
\newline
\textbf{Proof: } Suppose $f$ is not injective. 
Since $f: X \rightarrow Y$, we take two numbers $x_1, x_2 \in \mathbb{Z},$ where $x_1, x_2$ are in the set $X$ and $f(x_1) \text{ and } f(x_2)$ are in set $Y$, giving:
$$f(x_1) = f(x_2) \text{ when } x_1 \not = x_2,$$
Similarily, since $g: Y \rightarrow Z$, this would imply that:
$$(g \circ f)(x_1) = (g \circ f)(x_2) \text{ when } f(x_1) \not = f(x_2) \text{ ie, }$$
$$g(f(x_1)) = g(f(x_2)) \text{ when } f(x_1) \not = f(x_2).$$
\newline
Since $f(x_1), f(x_2) \in Y$ and $g(f(x_1)) = g(f(x_2)) \in Z$, we can consider that this proves the statement " if $f$ is not injective, then $g \circ f$ is not injective".
\newline
\newline
Therefore, by contrapositive, we can conclude that if $g \circ f$ is injective then $f$ is injective.
\newpage
\subsection{c)}
\textbf{ If $g \circ f$ is injective and $f$ is surjective, then $g$ is injective}
\newline
\newline
\textbf{Proof }
Consider two variables $y_1, y_2 \in  Y.$  $\text{ such that } g(y_1) = g(y_2); \text{ where } y_1, y_2 \in \mathbb{R}$
\newline
\newline
Since $f$ is known to be surjective, we can consider two other variables $x_1, x_2 \in X; \text{ where } x_1, x_2 \in \mathbb{R}.$
\newline
\newline
Then, if we map $f$ to $g$, using this surjective nature of $f$, we can presume $f(x_1) = y_1, f(x_2) = y_2.$ With this, the proof follows:
$$\Rightarrow g(f(x_1)) = g(f(x_2)),$$
$$\Rightarrow g \circ f(x_1) = g \circ f(x_2),$$
where $x_1$ = $x_2$ because $g \circ f$ is injective (given in question).
\newline
Then,
$$\Rightarrow f(x_1) = f(x_2),$$
$$\Rightarrow y_1 = y_2.$$
Thus, $g(y_1) = g(y_2) \Rightarrow y_1 = y_2,$ which means $g$ is injective. 
\newline
\newline
Therefore, we can conclude that if $g \circ f$ is injective and $f$ is surjective, then $g$ is injective.
\end{document}