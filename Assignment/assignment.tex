\documentclass{article}
\usepackage[english]{babel}
\usepackage[letterpaper,top=2cm,bottom=2cm,left=3cm,right=3cm,marginparwidth=1.75cm]{geometry}
\usepackage{amsmath}
\usepackage{amssymb}
\usepackage{graphicx}
\usepackage[colorlinks=true, allcolors=black]{hyperref}

\title{Maths Assignment - Applied Mathematics Flavour}
\author{z5417727}

\begin{document}
\maketitle
{\LARGE \tableofcontents}
\newpage
\section{Question 1}
\subsection{a)}
\textbf{ Prove that in modulo $9$, it is not possible for a perfect square to be congruent to $2, 3, 4, 6 \text{ or } 8$.}
\textbf{ Proof: } For any integer $n \in \mathbb{Z}$, we say that $n^2 \equiv 0, 1, 4, 7 (\text{ mod } 9).$ 
\newline
\newline
This can be deduced by finding the squares of $0, 1, 2, 3, 4$ respectively and applying the [] theorem for numbers up to $9$.
$$0^2 \equiv 0 (\text{mod }9),$$
$$1^2 \equiv 1 (\text{mod }9),$$
$$2^2 \equiv 4 (\text{mod }9),$$
$$3^2 \equiv 0 (\text{mod }9),$$
$$4^2 \equiv 7 (\text{mod }9).$$

Through [], we find the similar rule applied to $5$ through $8$ (since $9^2 \equiv 0 (\text{ mod } 9)$).
$$5^2 \equiv (-4)^2 \equiv 7 (\text{mod }9),$$
$$6^2 \equiv (-3)^2 \equiv 0 (\text{mod }9),$$
$$7^2 \equiv (-2)^2 \equiv 4 (\text{mod }9),$$
$$8^2 \equiv (-1)^2 \equiv 1 (\text{mod }9).$$
Here, we see that the modulo of perfect squares always end with the digits $0, 1, 4$ and $7$. Thus, it can be proved that in modulo $9$, it is not possible for a perfect square to be congruent to $2, 3, 4, 6, \text{ or } 8$.
\newpage
\subsection{b)}ƒƒ
\textbf{ Hence (and not otherwise) prove that there do not exist three consecutive integer values of $n$ for which $41n + 39$ is a perfect square. }
Consider a number $n$, $n + 1$ and $n + 2$ for $n \in \mathbb{R}$.
\section{Question 2}
\textbf{ A certain relation $\star$ is defined on the set $\mathbb{Z}^+$ by:
\newline
$$x \star y \text{ if and only if every factor of } x \text{ is a factor of } y.$$
For each of the questions below, be sure to provide a proof supporting your answer. }
\subsection{a)}
\textbf{ Is $\star $ reflexive? }
\newpage
\subsection{b)}
\textbf{Is $\star $ symmetric?}
\newpage
\subsection{c)}
\textbf{Is $\star $ anti-symmetric? }
\newpage
\subsection{d)}
\textbf{ Is $\star $ transitive?}
\newpage
\subsection{e)}
\textbf{ Is $\star $ an equivalence relation, a partial order, both or neither?}
\newpage
\section{Question 3}
\textbf{ Consider the two functions $f: X \rightarrow Y$ and $g: Y \rightarrow Z$ for non-empty sets $X, Y, Z$.
Decide whether each of the following statements is true or false, and prove each claim.}
\subsection{a)}
\textbf{ If $g \circ f$ is injective, then $g$ is injective.}
\newpage
\subsection{b)}
\textbf{ If $g \circ f$ is injective, then $f$ is injective. }
\newpage
\subsection{c)}
\textbf{ If $g \circ f$ is injective and $f$ is surjective, then $g$ is injective}
\end{document}